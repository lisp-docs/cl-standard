% -*- Mode: TeX -*-

\def\SatisfyTest#1#2{The \param{test}, \param{test-not}, and \param{key} 
affect how it is determined whether \param{#1} is the same as #2.
For details, \seesection\SatisfyingTheTwoArgTest.\par}

% Cons
%  Trees
%  Lists
%   List Elements
%   List Tails
%   List Mapping
%   Alists
%   Plists
%   Sets

%-------------------- Conses, lists, atoms, ... --------------------

\begincom{list}\ftype{System Class}

\label Class Precedence List::

\typeref{list},
\typeref{sequence},
\typeref{t}

\label Description::

A \newterm{list} is a chain of \term{conses} in which the \term{car} of each
\term{cons} is an \term{element} of the \term{list}, and the \term{cdr} of
each \term{cons} is either the next link in the chain or a terminating
\term{atom}.

A \newterm{proper list} is a chain of \term{conses} terminated by 
the \newterm{empty list}, \empty, which is  itself a \term{proper list}.
A \newterm{dotted list} is a \term{list} which has a terminating \term{atom} 
that is not the \term{empty list}.
A \newterm{circular list} is a chain of \term{conses} that has no termination
because some \term{cons} in the chain is the \term{cdr} of a later \term{cons}.

\term{Dotted lists} and \term{circular lists} are also \term{lists}, but usually
the unqualified term ``list'' within this specification means \term{proper list}.
Nevertheless, \thetype{list} unambiguously includes \term{dotted lists} 
and \term{circular lists}.

For each \term{element} of a \term{list} there is a \term{cons}.
The \term{empty list} has no \term{elements} and is not a \term{cons}.

%% 2.15.0 14
\Thetypes{cons} and \typeref{null} form an \term{exhaustive partition} 
of the \term{type} \typeref{list}.

\label See Also::

{\secref\LeftParen},
{\secref\PrintingListsAndConses}

\endcom%{list}\ftype{System Class}

\begincom{null}\ftype{System Class}

\label Class Precedence List::
\typeref{null},
\typeref{symbol},
\typeref{list},
\typeref{sequence},
\typeref{t}

\label Description::

%% 2.15.0 13
The only \term{object} \oftype{null} is \nil, 
which represents the \term{empty list} and can also be notated \empty.

\label See Also::

{\secref\SymbolTokens},
{\secref\LeftParen},
{\secref\PrintingSymbols}

\endcom%{null}\ftype{System Class}
\begincom{cons}\ftype{System Class}

\label Class Precedence List::
\typeref{cons},
\typeref{list},
\typeref{sequence},
\typeref{t}

\label Description::

A \term{cons} is a compound \term{object} having two components,
called the \term{car} and \term{cdr}. These form a \term{dotted pair}.
Each component can be any \term{object}.

\issue{CONS-TYPE-SPECIFIER:ADD}

\label Compound Type Specifier Kind::

Specializing.

\label Compound Type Specifier Syntax::

\Deftype{cons}{\ttbrac{car-typespec \brac{cdr-typespec}}}

\label Compound Type Specifier Arguments::

\param{car-typespec}---a \term{type specifier},
		    or the \term{symbol} \misc{*}.
  \Default{the \term{symbol} \misc{*}}

\param{cdr-typespec}---a \term{type specifier},
		    or the \term{symbol} \misc{*}.
  \Default{the \term{symbol} \misc{*}}

\label Compound Type Specifier Description::

This denotes the set of \term{conses} 
whose \term{car} is constrained to be of \term{type} \param{car-typespec} and
whose \term{cdr} is constrained to be of \term{type} \param{cdr-typespec}.
(If either \param{car-typespec} or \param{cdr-typespec} is \misc{*},
 it is as if \thetype{t} had been denoted.)

\endissue{CONS-TYPE-SPECIFIER:ADD}

\label See Also::

{\secref\LeftParen},
{\secref\PrintingListsAndConses}

\endcom%{cons}\ftype{System Class}
\begincom{atom}\ftype{Type}

%Added per Barrett's suggestion. -kmp 23-Oct-90

\label Supertypes:: 

\typeref{atom},
\typeref{t}

\label Description::

It is equivalent to {\tt (not cons)}.  

\endcom%{atom}\ftype{Type}


%-------------------- Cons --------------------

%%% ========== CONS
\begincom{cons}\ftype{Function}

\label Syntax::

\DefunWithValues cons {object-1 object-2} {cons}

\label Arguments and Values::

\param{object-1}---an \term{object}.

\param{object-2}---an \term{object}.

\param{cons}---a \term{cons}.

\label Description::

%% 15.1.0 8
Creates a \term{fresh} \term{cons}, the \term{car} of which is \param{object-1}
and the \term{cdr} of which is \param{object-2}.

\label Examples::

\code
 (cons 1 2) \EV (1 . 2)
 (cons 1 nil) \EV (1)
 (cons nil 2) \EV (NIL . 2)
 (cons nil nil) \EV (NIL)
 (cons 1 (cons 2 (cons 3 (cons 4 nil)))) \EV (1 2 3 4)
 (cons 'a 'b) \EV (A . B)
 (cons 'a (cons 'b (cons 'c '\empty))) \EV (A B C)
 (cons 'a '(b c d)) \EV (A B C D)
\endcode

\label Side Effects:\None.

%% Sandra thinks this is excessive.
%Creates an \term{object}.

\label Affected By:\None.

\label Exceptional Situations:\None.

%% Sandra thinks this is excessive.
%Might signal \typeref{storage-condition}.

\label See Also::

\funref{list}

\label Notes::
If \param{object-2} is a \term{list}, \funref{cons} can be thought of as
producing a new \term{list} which is like it but has \param{object-1} prepended.
\endcom


%%% ========== CONSP
\begincom{consp}\ftype{Function}

\label Syntax::

\DefunWithValues consp {object} {generalized-boolean}

\label Arguments and Values::

\param{object}---an \term{object}.

\param{generalized-boolean}---a \term{generalized boolean}.

\label Description::

%% 6.2.2 8
\TypePredicate{object}{cons}

\label Examples::
\code
 (consp nil) \EV \term{false}
 (consp (cons 1 2)) \EV \term{true}
\endcode

The \term{empty list} is not a \term{cons}, so

\code
 (consp '()) \EQ (consp 'nil) \EV \term{false}
\endcode

\label Side Effects:\None!

\label Affected By:\None!

\label Exceptional Situations:\None!

\label See Also::

\funref{listp}

\label Notes::

\code
 (consp \param{object}) \EQ (typep \param{object} 'cons) \EQ (not (typep \param{object} 'atom)) \EQ (typep \param{object} '(not atom))
\endcode

\endcom

%%% ========== ATOM
\begincom{atom}\ftype{Function}

\label Syntax::

\DefunWithValues atom {object} {generalized-boolean}

\label Arguments and Values::

\param{object}---an \term{object}.

\param{generalized-boolean}---a \term{generalized boolean}.

\label Description::

%% 6.2.2 6
\TypePredicate{object}{atom}

\label Examples::
\code
 (atom 'sss) \EV \term{true}
 (atom (cons 1 2)) \EV \term{false}
 (atom nil) \EV \term{true}
 (atom '()) \EV \term{true}
 (atom 3) \EV \term{true}
\endcode

\label Affected By:\None!

\label Exceptional Situations:\None!

\label See Also:\None.

\label Notes::

\code
 (atom \param{object}) \EQ (typep \param{object} 'atom) \EQ (not (consp \param{object}))
 \EQ (not (typep \param{object} 'cons)) \EQ (typep \param{object} '(not cons))
\endcode

\endcom

%%% ========== RPLACA
%%% ========== RPLACD
\begincom{rplaca, rplacd}\ftype{Function}

\label Syntax::

\DefunMultiWithValues {cons object} {cons}
 {\entry{rplaca}
  \entry{rplacd}}

\label Pronunciation::

\funref{rplaca}: \pronounced{\stress{r\harde}\Stress{plak}\schwa}
	      or \pronounced{\stress{r\schwa}\Stress{plak}\schwa}

\funref{rplacd}: \pronounced{\stress{r\harde}\Stress{plak}d\schwa}
	      or \pronounced{\stress{r\schwa}\Stress{plak}d\schwa}
	      or \pronounced{\stress{r\harde}\Stress{plak}d\harde}
	      or \pronounced{\stress{r\schwa}\Stress{plak}d\harde}

\label Arguments and Values::

\param{cons}---a \term{cons}.

\param{object}---an \term{object}.

\label Description::

%% 15.3.0 4
\funref{rplaca} replaces the \term{car} of the \param{cons} with \param{object}.
%and then returns the modified \param{cons}.

\funref{rplacd} replaces the \term{cdr} of the \param{cons} with \param{object}.
%and then returns the modified \param{cons}.

\label Examples::
%% 15.3.0 3
\code
 (defparameter *some-list* (list* 'one 'two 'three 'four)) \EV *some-list*
 *some-list* \EV (ONE TWO THREE . FOUR)
 (rplaca *some-list* 'uno) \EV (UNO TWO THREE . FOUR)
 *some-list* \EV (UNO TWO THREE . FOUR)
 (rplacd (last *some-list*) (list 'IV)) \EV (THREE IV)
 *some-list* \EV (UNO TWO THREE IV)
\endcode

\label Side Effects::
%% 15.3.0 1
%% 15.3.0 2
                   
The \param{cons} is modified. 

\label Affected By:\None!

\label Exceptional Situations:\None.

%!!! Issue: read-only structure might be enforced. -kmp

\Shouldchecktype{cons}{a \term{cons}}

\label See Also:\None.

\label Notes:\None.

%%What else would they be used for? -kmp 15-Feb-91
% \funref{rplaca} and \funref{rplacd} may be used to make alterations 
% in an already existing list structure, that is, 
% to change the \term{car} or \term{cdr} of an existing \term{cons}.

\endcom

%%% ========== CAR
%%% ========== CDR
%%% ========== CAAR
%%% ========== CADR
%%% ========== CDAR
%%% ========== CDDR
%%% ========== CAAAR
%%% ========== CAADR
%%% ========== CADAR
%%% ========== CADDR
%%% ========== CDAAR
%%% ========== CDADR
%%% ========== CDDAR
%%% ========== CDDDR
%%% ========== CAAAAR
%%% ========== CAAADR
%%% ========== CAADAR
%%% ========== CAADDR
%%% ========== CADAAR
%%% ========== CADADR
%%% ========== CADDAR
%%% ========== CADDDR
%%% ========== CDAAAR
%%% ========== CDAADR
%%% ========== CDADAR
%%% ========== CDADDR
%%% ========== CDDAAR
%%% ========== CDDADR
%%% ========== CDDDAR 
%%% ========== CDDDDR 
\begincom{car,    cdr,
          caar,   cadr,   cdar,   cddr,
	  caaar,  caadr,  cadar,  caddr,  cdaar,  cdadr,  cddar,  cdddr,
          caaaar, caaadr, caadar, caaddr, cadaar, cadadr, caddar, cadddr,
          cdaaar, cdaadr, cdadar, cdaddr, cddaar, cddadr, cdddar, cddddr}\ftype{Accessor}

\label Syntax::

\DefunMultiAccessorWithValues {x} {object} {new-object}
 {\entry{car}
  \entry{cdr}
  \blankline
  \entry{caar}
  \entry{cadr}
  \entry{cdar}
  \entry{cddr}
  \blankline
  \entry{caaar}
  \entry{caadr}
  \entry{cadar}
  \entry{caddr}
  \entry{cdaar}
  \entry{cdadr}
  \entry{cddar}
  \entry{cdddr}
  \blankline
  \entry{caaaar}
  \entry{caaadr}
  \entry{caadar}
  \entry{caaddr}
  \entry{cadaar}
  \entry{cadadr}
  \entry{caddar}
  \entry{cadddr}
  \entry{cdaaar}
  \entry{cdaadr}
  \entry{cdadar}
  \entry{cdaddr}
  \entry{cddaar}
  \entry{cddadr}
  \entry{cdddar}
  \entry{cddddr}}

\label Pronunciation::

\funref{cadr}: \pronounced{\Stress{ka}\stress{d\schwa r}}

\funref{caddr}: \pronounced{\Stress{kad}\schwa \stress{d\schwa r}}
  	     or \pronounced{\Stress{ka}\stress{d\.ud\schwa r}}

\funref{cdr}: \pronounced{\Stress{k\.u}\stress{d\schwa r}}

\funref{cddr}: \pronounced{\Stress{k\.ud}\schwa \stress{d\schwa r}}
  	    or \pronounced{\Stress{k\schwa}\stress{d\.ud\schwa r}}

\label Arguments and Values::

\param{x}---a \term{list}.

\param{object}---an \term{object}.

\param{new-object}---an \term{object}.

\label Description::

%% 15.1.0 3
If \param{x} is a \term{cons}, \funref{car} returns the \term{car} 
of that \term{cons}.  If \param{x} is \nil, \funref{car} returns \nil.

%% 15.1.0 4
If \param{x} is a \term{cons}, \funref{cdr} returns the \term{cdr} 
of that \term{cons}.  If \param{x} is \nil, \funref{cdr} returns \nil.

\term{Functions} are provided which perform compositions of up to four
\funref{car} and \funref{cdr} operations.  Their \term{names} consist of 
a \f{C}, followed by two, three, or four occurrences of \f{A} or \f{D}, 
and finally an \f{R}.  The series of \f{A}'s and \f{D}'s in each
\term{function}'s \term{name} is chosen to identify the series of 
\funref{car} and \funref{cdr} operations that is performed by the function.
The order in which the \f{A}'s and \f{D}'s appear is the inverse of the
order in which the corresponding operations are performed.  \Thenextfigure\ 
defines the relationships precisely.

\tablefigtwo{CAR and CDR variants}{This \term{place} $\ldots$}{Is equivalent to this \term{place} $\ldots$}{
\f{(caar \param{x})}&\f{(car (car \param{x}))}\cr
\f{(cadr \param{x})}&\f{(car (cdr \param{x}))}\cr
\f{(cdar \param{x})}&\f{(cdr (car \param{x}))}\cr
\f{(cddr \param{x})}&\f{(cdr (cdr \param{x}))}\cr
\f{(caaar \param{x})}&\f{(car (car (car \param{x})))}\cr
\f{(caadr \param{x})}&\f{(car (car (cdr \param{x})))}\cr
\f{(cadar \param{x})}&\f{(car (cdr (car \param{x})))}\cr
\f{(caddr \param{x})}&\f{(car (cdr (cdr \param{x})))}\cr
\f{(cdaar \param{x})}&\f{(cdr (car (car \param{x})))}\cr
\f{(cdadr \param{x})}&\f{(cdr (car (cdr \param{x})))}\cr
\f{(cddar \param{x})}&\f{(cdr (cdr (car \param{x})))}\cr
\f{(cdddr \param{x})}&\f{(cdr (cdr (cdr \param{x})))}\cr
\f{(caaaar \param{x})}&\f{(car (car (car (car \param{x}))))}\cr
\f{(caaadr \param{x})}&\f{(car (car (car (cdr \param{x}))))}\cr
\f{(caadar \param{x})}&\f{(car (car (cdr (car \param{x}))))}\cr
\f{(caaddr \param{x})}&\f{(car (car (cdr (cdr \param{x}))))}\cr
\f{(cadaar \param{x})}&\f{(car (cdr (car (car \param{x}))))}\cr
\f{(cadadr \param{x})}&\f{(car (cdr (car (cdr \param{x}))))}\cr
\f{(caddar \param{x})}&\f{(car (cdr (cdr (car \param{x}))))}\cr
\f{(cadddr \param{x})}&\f{(car (cdr (cdr (cdr \param{x}))))}\cr
\f{(cdaaar \param{x})}&\f{(cdr (car (car (car \param{x}))))}\cr
\f{(cdaadr \param{x})}&\f{(cdr (car (car (cdr \param{x}))))}\cr
\f{(cdadar \param{x})}&\f{(cdr (car (cdr (car \param{x}))))}\cr
\f{(cdaddr \param{x})}&\f{(cdr (car (cdr (cdr \param{x}))))}\cr
\f{(cddaar \param{x})}&\f{(cdr (cdr (car (car \param{x}))))}\cr
\f{(cddadr \param{x})}&\f{(cdr (cdr (car (cdr \param{x}))))}\cr
\f{(cdddar \param{x})}&\f{(cdr (cdr (cdr (car \param{x}))))}\cr
\f{(cddddr \param{x})}&\f{(cdr (cdr (cdr (cdr \param{x}))))}\cr
}

\macref{setf} can also be used with any of these functions to change an
existing component of \param{x}, but \macref{setf} will not make new
components.  So, for example, the \term{car} of a \term{cons} 
can be assigned with \macref{setf} of \funref{car},
but the \term{car} of \nil\ cannot be assigned with \macref{setf} of \funref{car}.
Similarly, the \term{car} of the \term{car} of a \term{cons} whose \term{car}
is a \term{cons} can be assigned with \macref{setf} of \funref{caar},
but neither \nil nor a \term{cons} whose car is \nil\ can be assigned
with \macref{setf} of \funref{caar}.

\issue{DOTTED-LIST-ARGUMENTS:CLARIFY}
The argument \param{x} is permitted to be a \term{dotted list} 
or a \term{circular list}.
\endissue{DOTTED-LIST-ARGUMENTS:CLARIFY}

\label Examples::

\code
 (car nil) \EV NIL  
 (cdr '(1 . 2)) \EV 2
 (cdr '(1 2)) \EV (2)
 (cadr '(1 2)) \EV 2 
 (car '(a b c)) \EV A
 (cdr '(a b c)) \EV (B C)
\endcode

\label Affected By:\None.

\label Exceptional Situations::

The functions \funref{car} and \funref{cdr} 
should signal \typeref{type-error} if they receive an argument which is not a
\term{list}.  The other functions (\funref{caar}, \funref{cadr},
$\ldots$ \funref{cddddr}) should behave for the purpose of
error checking as if defined by appropriate calls to \funref{car} and
\funref{cdr}.

\label See Also::

\funref{rplaca}, \funref{first}, \funref{rest}

\label Notes::

%% 15.1.0 7
The \term{car} of a \term{cons} can also be altered by using \funref{rplaca},
and the \term{cdr} of a \term{cons} can be altered by using \funref{rplacd}.

\code
(car \i{x})    \EQ (first \i{x})
(cadr \i{x})   \EQ (second \i{x}) \EQ (car (cdr \i{x}))
(caddr \i{x})  \EQ (third \i{x})  \EQ (car (cdr (cdr \i{x})))
(cadddr \i{x}) \EQ (fourth \i{x}) \EQ (car (cdr (cdr (cdr \i{x}))))
\endcode

\endcom

%-------------------- Trees --------------------

%%% ========== COPY-TREE
\begincom{copy-tree}\ftype{Function}

\issue{DOTTED-LIST-ARGUMENTS:CLARIFY}

\label Syntax::

\DefunWithValues copy-tree {tree} {new-tree}

\label Arguments and Values::

\param{tree}---a \term{tree}.

\param{new-tree}---a \term{tree}.

\label Description::

%% 15.2.0 25
Creates a \term{copy} of a \term{tree} of \term{conses}.

If \param{tree} is not a \term{cons}, it is returned; 
otherwise, the result is a new \term{cons} of the results of calling \funref{copy-tree} 
on the \term{car} and \term{cdr} of \param{tree}.
In other words, all \term{conses} in the \term{tree} represented by \param{tree}
are copied recursively, stopping only when non-\term{conses} are encountered.

\funref{copy-tree} does not preserve circularities and the sharing of substructure.
%!!! And consequently might fail to return in the case of circularity? -kmp 09-Apr-91

\label Examples::

\code
 (setq object (list (cons 1 "one")
                    (cons 2 (list 'a 'b 'c))))
\EV ((1 . "one") (2 A B C))
 (setq object-too object) \EV ((1 . "one") (2 A B C))
 (setq copy-as-list (copy-list object))
 (setq copy-as-alist (copy-alist object))
 (setq copy-as-tree (copy-tree object))
 (eq object object-too) \EV \term{true}
 (eq copy-as-tree object) \EV \term{false}
 (eql copy-as-tree object) \EV \term{false}
 (equal copy-as-tree object) \EV \term{true}
 (setf (first (cdr (second object))) "a"
       (car (second object)) "two"
       (car object) '(one . 1)) \EV (ONE . 1)
 object \EV ((ONE . 1) ("two" "a" B C))
 object-too \EV ((ONE . 1) ("two" "a" B C))
 copy-as-list \EV ((1 . "one") ("two" "a" B C))
 copy-as-alist \EV ((1 . "one") (2 "a" B C))
 copy-as-tree \EV ((1 . "one") (2 A B C)) 
\endcode

\label Side Effects:\None.

\label Affected By:\None.

\label Exceptional Situations:\None.

\label See Also::

\funref{tree-equal}

\label Notes:\None.

\endissue{DOTTED-LIST-ARGUMENTS:CLARIFY}

\endcom

%%% ========== NSUBLIS
%%% ========== SUBLIS
\begincom{sublis, nsublis}\ftype{Function}

\label Syntax::

\DefunWithValues sublis  {alist tree {\key} key test test-not} {new-tree}
\DefunWithValues nsublis {alist tree {\key} key test test-not} {new-tree}

\label Arguments and Values::

\param{alist}---an \term{association list}.

\issue{DOTTED-LIST-ARGUMENTS:CLARIFY}
\param{tree}---a \term{tree}.
\endissue{DOTTED-LIST-ARGUMENTS:CLARIFY}

\param{test}---a \term{designator} for a \term{function} of two \term{arguments}
  that returns a \term{generalized boolean}.

\param{test-not}---a \term{designator} for 
  a \term{function} of two \term{arguments}
  that returns a \term{generalized boolean}.

\param{key}---a \term{designator} for a \term{function} of one argument,
  or \nil.

\issue{DOTTED-LIST-ARGUMENTS:CLARIFY}
\param{new-tree}---a \term{tree}.
\endissue{DOTTED-LIST-ARGUMENTS:CLARIFY}

\label Description::

%% 15.4.0 9
\funref{sublis} makes substitutions for \term{objects} in \param{tree}
(a structure of \term{conses}).
%% 15.4.0 10
\funref{nsublis} is like \funref{sublis} 
but destructively modifies the relevant
parts of the \param{tree}.

\funref{sublis} looks at all subtrees and leaves of \param{tree};
if a subtree or leaf appears as a key in \param{alist}
(that is, the key and the subtree or leaf \term{satisfy the test}),
it is replaced by the \term{object} with which that key is associated.
This operation is non-destructive.  In effect, \funref{sublis} can
perform several \funref{subst} operations simultaneously.

%% Implied by "satisfy the test." -kmp 15-Feb-92
% The first argument to the \kwd{test} or \kwd{test-not}
% function is a key appearing in \param{alist};
% the second argument is
% the part of an element of \param{tree} extracted by the
% \kwd{key} function (if supplied).
% 
% The argument to the \kwd{key} 
% function is a \param{tree} element; the return value is part of
% the supplied \param{tree} element.
% If \kwd{key} is not supplied or \nil, 
% the \param{tree} element itself is 
% supplied to the \kwd{test} or \kwd{test-not}
% function.

If \funref{sublis} succeeds, a new copy of \param{tree} is returned in
which each occurrence of such a subtree or leaf is replaced by the
\term{object} with which it is associated.   If no changes are made, the
original tree is returned.  The original \param{tree} is left unchanged,
but the result tree may share cells with it.

\funref{nsublis} is permitted to modify \param{tree} 
but otherwise returns the same values as \funref{sublis}.

\label Examples::

\code
 (sublis '((x . 100) (z . zprime))
         '(plus x (minus g z x p) 4 . x))
\EV (PLUS 100 (MINUS G ZPRIME 100 P) 4 . 100)
 (sublis '(((+ x y) . (- x y)) ((- x y) . (+ x y)))
         '(* (/ (+ x y) (+ x p)) (- x y))
         :test #'equal)
\EV (* (/ (- X Y) (+ X P)) (+ X Y))
 (setq tree1 '(1 (1 2) ((1 2 3)) (((1 2 3 4)))))
\EV (1 (1 2) ((1 2 3)) (((1 2 3 4))))
 (sublis '((3 . "three")) tree1) 
\EV (1 (1 2) ((1 2 "three")) (((1 2 "three" 4))))
 (sublis '((t . "string"))
          (sublis '((1 . "") (4 . 44)) tree1)
          :key #'stringp)
\EV ("string" ("string" 2) (("string" 2 3)) ((("string" 2 3 44))))
 tree1 \EV (1 (1 2) ((1 2 3)) (((1 2 3 4))))
 (setq tree2 '("one" ("one" "two") (("one" "Two" "three"))))
\EV ("one" ("one" "two") (("one" "Two" "three"))) 
 (sublis '(("two" . 2)) tree2) 
\EV ("one" ("one" "two") (("one" "Two" "three"))) 
 tree2 \EV ("one" ("one" "two") (("one" "Two" "three"))) 
 (sublis '(("two" . 2)) tree2 :test 'equal) 
\EV ("one" ("one" 2) (("one" "Two" "three"))) 

 (nsublis '((t . 'temp))
           tree1
           :key #'(lambda (x) (or (atom x) (< (list-length x) 3))))
\EV ((QUOTE TEMP) (QUOTE TEMP) QUOTE TEMP) 
\endcode

\label Side Effects::

\funref{nsublis} modifies \param{tree}.
 
\label Affected By:\None.

\label Exceptional Situations:\None.

\label See Also::

\funref{subst},
\issue{CONSTANT-MODIFICATION:DISALLOW}
{\secref\ConstantModification},
\endissue{CONSTANT-MODIFICATION:DISALLOW}
\issue{MAPPING-DESTRUCTIVE-INTERACTION:EXPLICITLY-VAGUE}
{\secref\TraversalRules}
\endissue{MAPPING-DESTRUCTIVE-INTERACTION:EXPLICITLY-VAGUE}

\label Notes::

\issue{TEST-NOT-IF-NOT:FLUSH-ALL}
The \kwd{test-not} parameter is deprecated.
\endissue{TEST-NOT-IF-NOT:FLUSH-ALL}

Because the side-effecting variants (\eg \funref{nsublis}) potentially
change the path that is being traversed, their effects in the presence
of shared or circular structure structure may vary in surprising ways
when compared to their non-side-effecting alternatives.  To see this,
consider the following side-effect behavior, which might be exhibited by
some implementations:

\code
 (defun test-it (fn)
   (let* ((shared-piece (list 'a 'b))
          (data (list shared-piece shared-piece)))
     (funcall fn '((a . b) (b . a)) data)))
 (test-it #'sublis) \EV ((B A) (B A))
 (test-it #'nsublis) \EV ((A B) (A B))
\endcode


\endcom

%%% ========== NSUBST
%%% ========== NSUBST-IF
%%% ========== NSUBST-IF-NOT
%%% ========== SUBST
%%% ========== SUBST-IF
%%% ========== SUBST-IF-NOT
\begincom{subst, subst-if, subst-if-not, nsubst, nsubst-if, nsubst-if-not}\ftype{Function}

\label Syntax::

\DefunWithValues subst         {new old  tree {\key} key test test-not} {new-tree}
\DefunWithValues subst-if      {new predicate tree {\key} key} {new-tree}
\DefunWithValues subst-if-not  {new predicate tree {\key} key} {new-tree}

\DefunWithValues nsubst        {new old  tree {\key} key test test-not} {new-tree}
\DefunWithValues nsubst-if     {new predicate tree {\key} key} {new-tree}
\DefunWithValues nsubst-if-not {new predicate tree {\key} key} {new-tree}

\label Arguments and Values:: 

\param{new}---an \term{object}.

\param{old}---an \term{object}.

\param{predicate}---a \term{symbol} that names a \term{function},
   or a \term{function} of one argument 
      that returns a \term{generalized boolean} value.

\issue{DOTTED-LIST-ARGUMENTS:CLARIFY}
\param{tree}---a \term{tree}.
\endissue{DOTTED-LIST-ARGUMENTS:CLARIFY}

\param{test}---a \term{designator} for a \term{function} of two \term{arguments}
  that returns a \term{generalized boolean}.

\param{test-not}---a \term{designator} for 
  a \term{function} of two \term{arguments}
  that returns a \term{generalized boolean}.

\param{key}---a \term{designator} for a \term{function} of one argument,
  or \nil.

\issue{DOTTED-LIST-ARGUMENTS:CLARIFY}
\param{new-tree}---a \term{tree}.
\endissue{DOTTED-LIST-ARGUMENTS:CLARIFY}

\label Description::

\funref{subst}, \funref{subst-if}, and \funref{subst-if-not} perform
substitution operations on \param{tree}.  
Each function searches \param{tree} for occurrences of a 
particular \param{old} item of an element or subexpression that 
\term{satisfies the test}.

\funref{nsubst}, \funref{nsubst-if}, and \funref{nsubst-if-not} are 
like \funref{subst},
\funref{subst-if}, and \funref{subst-if-not} respectively, except that the 
original \param{tree} is  modified.

%% 15.4.0 4
\funref{subst} makes a copy of \param{tree},
substituting \param{new} for every subtree or leaf of \param{tree}
(whether the subtree or leaf is a \term{car} or a \term{cdr} of its parent)
such that \param{old} and the subtree or leaf \term{satisfy the test}.  

%% 15.4.0 7
\funref{nsubst} is a destructive version of \funref{subst}.  
The list structure of
\param{tree} is altered by destructively replacing with \param{new}
each leaf of the \param{tree} such that \param{old} and the leaf
\term{satisfy the test}.

%% Implied by "satisfy the test".
% Either \kwd{test} or \kwd{test-not} can
% be used to supply a test function other than the default.
% The first argument to the \kwd{test} or \kwd{test-not} function
% is \param{old};
% the second argument is 
% the part of an element of \param{tree} 
% extracted by the \kwd{key} function (if supplied).
% The argument to the \kwd{key} function is an
% element of \param{tree}; the return value is part of the supplied
% argument.
% If \kwd{key} is not supplied, the element from
% \param{tree} is supplied to the \kwd{test} or \kwd{test-not} function.

For \funref{subst}, \funref{subst-if}, 
and \funref{subst-if-not},
if the functions succeed, a new
copy of the tree is returned in which each  occurrence of such an
element is replaced by the
\param{new} element or subexpression.  If no changes are made, the original 
\param{tree} may be returned.
The original \param{tree} is left unchanged, but the result tree 
may share storage with it.           

For \funref{nsubst}, \funref{nsubst-if}, 
and \funref{nsubst-if-not}
the original \param{tree} is  modified and returned as the function result,
but the result may not be \funref{eq} to \param{tree}.
%!!! Barmar: Huh??? ^^^^^^^^^^^^^^^^^^^^^^^^^^^^^^^^^^^

\label Examples::

%% 15.4.0 6
\code
 (setq tree1 '(1 (1 2) (1 2 3) (1 2 3 4))) \EV (1 (1 2) (1 2 3) (1 2 3 4))
 (subst "two" 2 tree1) \EV (1 (1 "two") (1 "two" 3) (1 "two" 3 4))
 (subst "five" 5 tree1) \EV (1 (1 2) (1 2 3) (1 2 3 4))
 (eq tree1 (subst "five" 5 tree1)) \EV \term{implementation-dependent}
 (subst 'tempest 'hurricane
        '(shakespeare wrote (the hurricane)))
\EV (SHAKESPEARE WROTE (THE TEMPEST))
 (subst 'foo 'nil '(shakespeare wrote (twelfth night)))
\EV (SHAKESPEARE WROTE (TWELFTH NIGHT . FOO) . FOO)
 (subst '(a . cons) '(old . pair)
        '((old . spice) ((old . shoes) old . pair) (old . pair))
        :test #'equal)
\EV ((OLD . SPICE) ((OLD . SHOES) A . CONS) (A . CONS))

 (subst-if 5 #'listp tree1) \EV 5
 (subst-if-not '(x) #'consp tree1) 
\EV (1 X)

 tree1 \EV (1 (1 2) (1 2 3) (1 2 3 4))
 (nsubst 'x 3 tree1 :key #'(lambda (y) (and (listp y) (third y)))) 
\EV (1 (1 2) X X)
 tree1 \EV (1 (1 2) X X)
\endcode

\label Side Effects::

%% 15.4.0 7
\funref{nsubst}, \funref{nsubst-if}, and \funref{nsubst-if-not} 
might alter the \term{tree structure} of \param{tree}.

\label Affected By:\None.

\label Exceptional Situations:\None.

\label See Also::

\funref{substitute},
\funref{nsubstitute},
\issue{CONSTANT-MODIFICATION:DISALLOW}
{\secref\ConstantModification},
\endissue{CONSTANT-MODIFICATION:DISALLOW}
\issue{MAPPING-DESTRUCTIVE-INTERACTION:EXPLICITLY-VAGUE}
{\secref\TraversalRules}
\endissue{MAPPING-DESTRUCTIVE-INTERACTION:EXPLICITLY-VAGUE}

\label Notes::

\issue{TEST-NOT-IF-NOT:FLUSH-ALL}
The \kwd{test-not} parameter is deprecated.
\endissue{TEST-NOT-IF-NOT:FLUSH-ALL}

\issue{TEST-NOT-IF-NOT:FLUSH-ALL}
The functions \funref{subst-if-not} and \funref{nsubst-if-not} are deprecated.
\endissue{TEST-NOT-IF-NOT:FLUSH-ALL}

One possible definition of \funref{subst}:

%% Indentation fixed per Boyer/Kaufmann/Moore #13 (by X3J13 vote at May 4-5, 1994 meeting)
%% -kmp 9-May-94
\code
 (defun subst (old new tree &rest x &key test test-not key)
   (cond ((satisfies-the-test old tree :test test
                              :test-not test-not :key key)
          new)
         ((atom tree) tree)
         (t (let ((a (apply #'subst old new (car tree) x))
                  (d (apply #'subst old new (cdr tree) x)))
              (if (and (eql a (car tree))
                       (eql d (cdr tree)))
                  tree
                  (cons a d))))))
\endcode
\endcom

%%% ========== TREE-EQUAL
\begincom{tree-equal}\ftype{Function}

\issue{DOTTED-LIST-ARGUMENTS:CLARIFY}

\label Syntax::

\DefunWithValues tree-equal {tree-1 tree-2 {\key} test test-not} {generalized-boolean}

\label Arguments and Values:: 

\param{tree-1}---a \term{tree}.

\param{tree-2}---a \term{tree}.

\param{test}---a \term{designator} for a \term{function} of two \term{arguments}
  that returns a \term{generalized boolean}.

\param{test-not}---a \term{designator} for 
  a \term{function} of two \term{arguments}
  that returns a \term{generalized boolean}.

\param{generalized-boolean}---a \term{generalized boolean}.

\label Description::

%% 15.1.0 8
\funref{tree-equal} tests whether two trees are of the same shape
and have the same leaves.  
\funref{tree-equal} returns \term{true} if \param{tree-1} and \param{tree-2}  are 
both \term{atoms} and \term{satisfy the test},
or if they are both \term{conses} and
the \term{car} of \param{tree-1} is \funref{tree-equal} to
the \term{car} of \param{tree-2} and
the \term{cdr} of \param{tree-1} is \funref{tree-equal} to
the \term{cdr} of \param{tree-2}.  
Otherwise, \funref{tree-equal} returns \term{false}.

\funref{tree-equal} recursively compares \term{conses} but not any 
other \term{objects} that have components.  

The first argument to the \kwd{test} or \kwd{test-not} 
function is \param{tree-1} or a \term{car} or \term{cdr} of \param{tree-1};
the second argument is \param{tree-2} or a \term{car} 
or \term{cdr} of \param{tree-2}.

\label Examples::

\code
 (setq tree1 '(1 (1 2))
       tree2 '(1 (1 2))) \EV (1 (1 2))
 (tree-equal tree1 tree2) \EV \term{true}
 (eql tree1 tree2) \EV \term{false}
 (setq tree1 '('a ('b 'c))
       tree2 '('a ('b 'c))) \EV ('a ('b 'c)) 
\EV ((QUOTE A) ((QUOTE B) (QUOTE C)))
 (tree-equal tree1 tree2 :test 'eq) \EV \term{true}
\endcode

\label Side Effects:\None.

\label Affected By:\None.

\label Exceptional Situations::

The consequences are undefined 
if both \param{tree-1} and \param{tree-2} are circular.

\label See Also::

\funref{equal},
\issue{MAPPING-DESTRUCTIVE-INTERACTION:EXPLICITLY-VAGUE}
{\secref\TraversalRules}
\endissue{MAPPING-DESTRUCTIVE-INTERACTION:EXPLICITLY-VAGUE}

\label Notes::

\issue{TEST-NOT-IF-NOT:FLUSH-ALL}
The \kwd{test-not} parameter is deprecated.
\endissue{TEST-NOT-IF-NOT:FLUSH-ALL}

\endissue{DOTTED-LIST-ARGUMENTS:CLARIFY}

\endcom

%-------------------- List --------------------

%%% ========== COPY-LIST
\begincom{copy-list}\ftype{Function}

\label Syntax::

\DefunWithValues copy-list {list} {copy}

\label Arguments and Values::

\param{list}---a \term{proper list} or a \term{dotted list}.

\param{copy}---a \term{list}.

\label Description::

%% 15.2.0 23

Returns a \term{copy} of \param{list}.
If \param{list} is a \term{dotted list},
the resulting \term{list} will also be a \term{dotted list}.

Only the \term{list structure} of \param{list} is copied; 
the \term{elements} of the resulting list are 
the \term{same} as the corresponding \term{elements} of the given \param{list}.

\label Examples::    

\code
 (setq lst (list 1 (list 2 3))) \EV (1 (2 3))
 (setq slst lst) \EV (1 (2 3))
 (setq clst (copy-list lst)) \EV (1 (2 3))
 (eq slst lst) \EV \term{true}
 (eq clst lst) \EV \term{false}
 (equal clst lst) \EV \term{true}
 (rplaca lst "one") \EV ("one" (2 3))
 slst \EV ("one" (2 3))
 clst \EV (1 (2 3))
 (setf (caadr lst) "two") \EV "two"
 lst \EV ("one" ("two" 3))
 slst \EV ("one" ("two" 3))
 clst \EV (1 ("two" 3))
\endcode

\label Side Effects:\None.

\label Affected By:\None.

\label Exceptional Situations::

The consequences are undefined if \param{list} is a \term{circular list}.

\label See Also::

\funref{copy-alist},
\funref{copy-seq},
\funref{copy-tree}

\label Notes::

The copy created is \funref{equal} to \param{list}, but not \funref{eq}.

\endcom

%%% ========== LIST
%%% ========== LIST*
\begincom{list, list*}\ftype{Function}

\label Syntax::

\DefunWithValues {list}  {{\rest} objects} {list}
\DefunWithValues {list*} {{\rest} \plus{objects}} {result}

\label Arguments and Values:: 

\param{object}---an \term{object}.

\param{list}---a \term{list}.

\param{result}---an \term{object}.
%A (possibly dotted) list, if at least 2 args are given.

\label Description::

%% 15.2.0 17

\funref{list} returns a \term{list} containing the supplied \param{objects}.  

%% 15.2.0 18

\funref{list*} is like \funref{list} except that 
the last \term{argument} to \funref{list} becomes 
the \term{car} of the last \term{cons} constructed, while
the last \term{argument} to \funref{list*} becomes 
the \term{cdr} of the last \term{cons} constructed.
Hence, any given call to \funref{list*} always produces one fewer \term{conses}
than a call to \funref{list} with the same number of arguments.

If the last \term{argument} to \funref{list*} is a \term{list}, 
the effect is to construct a new \term{list} which is similar, but
which has additional elements added to the front corresponding to
the preceding \term{arguments} of \funref{list*}.

If \funref{list*} receives only one \param{object},
that \param{object} is returned, regardless of whether or not it is a \term{list}.

\label Examples::

\code
 (list 1) \EV (1)
 (list* 1) \EV 1
 (setq a 1) \EV 1
 (list a 2) \EV (1 2)
 '(a 2) \EV (A 2)
 (list 'a 2) \EV (A 2)
 (list* a 2) \EV (1 . 2)
 (list) \EV NIL ;\ie ()
 (setq a '(1 2)) \EV (1 2)
 (eq a (list* a)) \EV \term{true}
 (list 3 4 'a (car '(b . c)) (+ 6 -2)) \EV (3 4 A B 4)
 (list* 'a 'b 'c 'd) \EQ (cons 'a (cons 'b (cons 'c 'd))) \EV (A B C . D)
 (list* 'a 'b 'c '(d e f)) \EV (A B C D E F)
\endcode

\label Side Effects:\None.

\label Affected By:\None.

\label Exceptional Situations:\None.

\label See Also::

\funref{cons}

\label Notes::

\code
 (list* \param{x}) \EQ \param{x}
\endcode

\endcom

%%% ========== LIST-LENGTH
\begincom{list-length}\ftype{Function}

\label Syntax::

\DefunWithValues list-length {list} {length}

\label Arguments and Values::

\param{list}---a \term{proper list} or a \term{circular list}.

\param{length}---a non-negative \term{integer}, or \nil.

\label Description::

%% 15.2.0 5
Returns the \term{length} of \param{list} if \param{list} is a \term{proper list}.
Returns \nil\ if \param{list} is a \term{circular list}.

\label Examples::

\code
 (list-length '(a b c d)) \EV 4
 (list-length '(a (b c) d)) \EV 3
 (list-length '()) \EV 0
 (list-length nil) \EV 0
 (defun circular-list (&rest elements)
   (let ((cycle (copy-list elements))) 
     (nconc cycle cycle)))
 (list-length (circular-list 'a 'b)) \EV NIL
 (list-length (circular-list 'a)) \EV NIL
 (list-length (circular-list)) \EV 0
\endcode

\label Side Effects:\None.

\label Affected By:\None.

\label Exceptional Situations::

\Shouldchecktype{list}{a \term{proper list} or a \term{circular list}}

\label See Also::

\funref{length}

\label Notes::

\funref{list-length} could be implemented as follows:

\code
 (defun list-length (x)  
   (do ((n 0 (+ n 2))           ;Counter.
        (fast x (cddr fast))    ;Fast pointer: leaps by 2.
        (slow x (cdr slow)))    ;Slow pointer: leaps by 1.
       (nil)
     ;; If fast pointer hits the end, return the count.
     (when (endp fast) (return n))
     (when (endp (cdr fast)) (return (+ n 1)))
     ;; If fast pointer eventually equals slow pointer,
     ;;  then we must be stuck in a circular list.
     ;; (A deeper property is the converse: if we are
     ;;  stuck in a circular list, then eventually the
     ;;  fast pointer will equal the slow pointer.
     ;;  That fact justifies this implementation.)
     (when (and (eq fast slow) (> n 0)) (return nil))))
 
\endcode

\endcom

%%% ========== LISTP
\begincom{listp}\ftype{Function}

\label Syntax::

\DefunWithValues listp {object} {generalized-boolean}

\label Arguments and Values::

\param{object}---an \term{object}.

\param{generalized-boolean}---a \term{generalized boolean}.

\label Description::

%% 6.2.2 10
\TypePredicate{object}{list}

\label Examples::
\code
 (listp nil) \EV \term{true}
 (listp (cons 1 2)) \EV \term{true}
 (listp (make-array 6)) \EV \term{false}
 (listp t) \EV \term{false}
\endcode

\label Side Effects:\None.

\label Affected By:\None.

\label Exceptional Situations:\None!

\label See Also::

\funref{consp}

\label Notes::

%This is implied by type list, but still worth saying. -kmp

If \param{object} is a \term{cons},
\funref{listp} does not check whether \param{object} is a \term{proper list};
it returns \term{true} for any kind of \term{list}.

\code
 (listp \param{object}) \EQ (typep \param{object} 'list) \EQ (typep \param{object} '(or cons null))
\endcode

\endcom

%%% ========== MAKE-LIST
\begincom{make-list}\ftype{Function}

\label Syntax::

\DefunWithValues make-list {size {\key} initial-element} {list}
                                   
\label Arguments and Values::

\param{size}---a non-negative \term{integer}.

\param{initial-element}---an \term{object}.
  \Default{\nil}

\param{list}---a \term{list}.

\label Description::

%% 15.2.0 19
Returns a \term{list} of \param{length} given by \term{size},
each of the \term{elements} of which is \param{initial-element}.

\label Examples::
\code
 (make-list 5) \EV (NIL NIL NIL NIL NIL)
 (make-list 3 :initial-element 'rah) \EV (RAH RAH RAH)
 (make-list 2 :initial-element '(1 2 3)) \EV ((1 2 3) (1 2 3))
 (make-list 0) \EV NIL ;\ie ()
 (make-list 0 :initial-element 'new-element) \EV NIL 
\endcode

\label Side Effects:\None.

\label Affected By:\None.

\label Exceptional Situations::

\Shouldchecktype{size}{a non-negative \term{integer}}

\label See Also::

\funref{cons},
\funref{list}

\label Notes:\None.

\endcom

%-------------------- List Elements --------------------

%%% ========== PUSH
\begincom{push}\ftype{Macro}

\label Syntax::

\DefmacWithValues push {item place} {new-place-value}

\label Arguments and Values::

\param{item}---an \term{object}. 

\issue{DOTTED-LIST-ARGUMENTS:CLARIFY}
\param{place}---a \term{place}, the \term{value} of which may be any \term{object}.  
\endissue{DOTTED-LIST-ARGUMENTS:CLARIFY}

\param{new-place-value}---a \term{list} (the new \term{value} of \param{place}).

\label Description::

\macref{push} prepends \param{item} to the \term{list} that is stored
in \param{place}, stores the resulting \term{list} in \param{place}, 
and returns the \term{list}.

\issue{PUSH-EVALUATION-ORDER:FIRST-ITEM}
For information about the \term{evaluation} of \term{subforms} of \param{place},
\seesection\GenRefSubFormEval.
\endissue{PUSH-EVALUATION-ORDER:FIRST-ITEM}

\label Examples::
%% 15.2.0 31
\code
 (setq llst '(nil)) \EV (NIL)
 (push 1 (car llst)) \EV (1)
 llst \EV ((1))
 (push 1 (car llst)) \EV (1 1)
 llst \EV ((1 1))
 (setq x '(a (b c) d)) \EV (A (B C) D)
 (push 5 (cadr x)) \EV (5 B C)  
 x \EV (A (5 B C) D)
\endcode

\label Side Effects::

The contents of \param{place} are modified.

\label Affected By:\None.

\label Exceptional Situations:\None.

\label See Also::

\macref{pop},
\macref{pushnew},
{\secref\GeneralizedReference}

\label Notes::
The effect of \f{(push \i{item} \i{place})}
is equivalent to

\code
 (setf place (cons \i{item} \i{place}))
\endcode
\issue{PUSH-EVALUATION-ORDER:FIRST-ITEM}
except that the \term{subforms} of \param{place} 
are evaluated only once, and \param{item} is evaluated
before \param{place}.
\endissue{PUSH-EVALUATION-ORDER:FIRST-ITEM}

\endcom

%%% ========== POP
\begincom{pop}\ftype{Macro}

\label Syntax::

\DefmacWithValues pop {place} {element}

\label Arguments and Values:: 

\issue{DOTTED-LIST-ARGUMENTS:CLARIFY}
\param{place}---a \term{place}, the \term{value} of which is a \term{list}
   (possibly, but necessarily, a \term{dotted list} or \term{circular list}).
\endissue{DOTTED-LIST-ARGUMENTS:CLARIFY}

\param{element}---an \term{object} (the \term{car} of the contents of \param{place}).

\label Description::

%% 15.2.0 35
\macref{pop} \term{reads} the \term{value} of \param{place},
remembers the \term{car} of the \term{list} which was retrieved,
\term{writes} the \term{cdr} of the \term{list} back into the \param{place},
and finally \term{yields} the \term{car} of the originally retrieved \term{list}.

\issue{PUSH-EVALUATION-ORDER:FIRST-ITEM}
For information about the \term{evaluation} of \term{subforms} of \param{place},
\seesection\GenRefSubFormEval.
\endissue{PUSH-EVALUATION-ORDER:FIRST-ITEM}

\label Examples::

\code
 (setq stack '(a b c)) \EV (A B C)
 (pop stack) \EV A  
 stack \EV (B C)
 (setq llst '((1 2 3 4))) \EV ((1 2 3 4))
 (pop (car llst)) \EV 1
 llst \EV ((2 3 4))
\endcode

\label Side Effects::

The contents of \param{place} are modified.

\label Affected By:\None.

\label Exceptional Situations:\None.

\label See Also::

\macref{push},
\macref{pushnew},
{\secref\GeneralizedReference}

\label Notes::

The effect of \f{(pop \param{place})} is roughly equivalent to

\code
 (prog1 (car \param{place}) (setf \param{place} (cdr \param{place})))
\endcode
except that the latter would evaluate any \term{subforms} of \param{place}
three times, while \macref{pop} evaluates them only once.


\endcom


%%% ========== FIRST
%%% ========== SECOND
%%% ========== THIRD
%%% ========== FOURTH
%%% ========== FIFTH
%%% ========== SIXTH
%%% ========== SEVENTH
%%% ========== EIGHTH
%%% ========== NINTH
%%% ========== TENTH
\begincom{first, second, third, fourth, fifth, 
	  sixth, seventh, eighth, ninth, tenth}\ftype{Accessor}
         
\label Syntax::

\DefunMultiAccessorWithValues {list} {object} {new-object}
 {\entry{first}
  \entry{second}
  \entry{third}
  \entry{fourth}
  \entry{fifth}
  \entry{sixth}
  \entry{seventh}
  \entry{eighth}
  \entry{ninth}
  \entry{tenth}}

\label Arguments and Values::

\param{list}---a \term{list}, 
\issue{DOTTED-LIST-ARGUMENTS:CLARIFY}
	       which might be a \term{dotted list} or a \term{circular list}.
\endissue{DOTTED-LIST-ARGUMENTS:CLARIFY}

\param{object}, \param{new-object}---an \param{object}.

\label Description::

The functions
%% 15.2.0 11
\funref{first},
\funref{second},
\funref{third}, 
\funref{fourth}, 
\funref{fifth},
\funref{sixth},
\funref{seventh},
\funref{eighth}, 
\funref{ninth},
and
\funref{tenth}
%% 15.2.0 12
\param{access} the first, second, third, fourth, fifth, sixth, seventh, eighth,
ninth, and tenth \term{elements} of \param{list}, respectively.
%% I changed this to be in terms of car/cdr because I felt this made 
%% the error checking requirements more apparent. -kmp 27-Aug-93
% If there is no such \term{element} during a \term{read}, \nil\ is returned.
% The consequences are undefined if there is no such \term{element} during a \term{write}.
Specifically,

\code
 (first \param{list})    \EQ  (car \param{list})
 (second \param{list})   \EQ  (car (cdr \param{list}))
 (third \param{list})    \EQ  (car (cddr \param{list}))
 (fourth \param{list})   \EQ  (car (cdddr \param{list}))
 (fifth \param{list})    \EQ  (car (cddddr \param{list}))
 (sixth \param{list})    \EQ  (car (cdr (cddddr \param{list})))
 (seventh \param{list})  \EQ  (car (cddr (cddddr \param{list})))
 (eighth \param{list})   \EQ  (car (cdddr (cddddr \param{list})))
 (ninth \param{list})    \EQ  (car (cddddr (cddddr \param{list})))
 (tenth \param{list})    \EQ  (car (cdr (cddddr (cddddr \param{list}))))
\endcode

\macref{setf} can also be used with any of these functions to change an
existing component.  The same equivalences apply.  For example:

\code
 (setf (fifth \param{list}) \param{new-object}) \EQ (setf (car (cddddr \param{list})) \param{new-object})
\endcode

\label Examples::

\code
 (setq lst '(1 2 3 (4 5 6) ((V)) vi 7 8 9 10)) 
\EV (1 2 3 (4 5 6) ((V)) VI 7 8 9 10)
 (first lst) \EV 1
 (tenth lst) \EV 10
 (fifth lst) \EV ((V))
 (second (fourth lst)) \EV 5
 (sixth '(1 2 3)) \EV NIL
 (setf (fourth lst) "four") \EV "four"
 lst \EV (1 2 3 "four" ((V)) VI 7 8 9 10)
\endcode

\label Side Effects:\None.

\label Affected By:\None.

\label Exceptional Situations:\None.

\label See Also::

\funref{car}, \funref{nth}

\label Notes::

\funref{first}  is functionally equivalent to \funref{car},
\funref{second} is functionally equivalent to \funref{cadr},
\funref{third}  is functionally equivalent to \funref{caddr}, and 
\funref{fourth} is functionally equivalent to \funref{cadddr}.

The ordinal numbering used here is one-origin,
as opposed to the zero-origin numbering used by \funref{nth}:

\code
 (fifth x) \EQ (nth 4 x)
\endcode

\endcom

%%% ========== NTH
\begincom{nth}\ftype{Accessor}

\label Syntax::

\DefunWithValues nth {n list} {object}
\Defsetf         nth {n list} {new-object}

\label Arguments and Values::

\param{n}---a non-negative \term{integer}.

\param{list}---a \term{list},
\issue{DOTTED-LIST-ARGUMENTS:CLARIFY}
	       which might be a \term{dotted list} or a \term{circular list}.
\endissue{DOTTED-LIST-ARGUMENTS:CLARIFY}

\param{object}---an \term{object}.

\param{new-object}---an \term{object}.

\label Description::

\funref{nth} locates the \param{n}th element of \param{list},
where the \term{car} of the \param{list} is the ``zeroth'' element.
\issue{DOTTED-LIST-ARGUMENTS:CLARIFY}
% If the length of \param{list} is not greater than \param{n}, 
% then the result is \nil.
Specifically,

\code
 (nth \param{n} \param{list}) \EQ (car (nthcdr \param{n} \param{list}))
\endcode
\endissue{DOTTED-LIST-ARGUMENTS:CLARIFY}

%% 15.2.0 8

\funref{nth} may be used to specify a \param{place} to \macref{setf}.
\issue{DOTTED-LIST-ARGUMENTS:CLARIFY}
Specifically,
% when \funref{nth} is used in this way, \param{n} must be less
% than the length of the \param{list}.

\code
 (setf (nth \param{n} \param{list}) \param{new-object}) \EQ (setf (car (nthcdr \param{n} \param{list})) \param{new-object})
\endcode
\endissue{DOTTED-LIST-ARGUMENTS:CLARIFY}

\label Examples::

%% 15.2.0 6
\code
 (nth 0 '(foo bar baz)) \EV FOO
 (nth 1 '(foo bar baz)) \EV BAR
 (nth 3 '(foo bar baz)) \EV NIL
 (setq 0-to-3 (list 0 1 2 3)) \EV (0 1 2 3)
 (setf (nth 2 0-to-3) "two") \EV "two"
 0-to-3 \EV (0 1 "two" 3)
\endcode

\label Side Effects:\None.

\label Affected By:\None.

\label Exceptional Situations:\None.

\label See Also::

\funref{elt},
\funref{first},
\funref{nthcdr}

\label Notes:\None.

\endcom

%-------------------- List Tails --------------------

%%% ========== ENDP
\begincom{endp}\ftype{Function}

\label Syntax::

\DefunWithValues endp {list} {generalized-boolean}

\label Arguments and Values::

\param{list}---a \term{list},
\issue{DOTTED-LIST-ARGUMENTS:CLARIFY}
	       which might be a \term{dotted list} or a \term{circular list}.
\endissue{DOTTED-LIST-ARGUMENTS:CLARIFY}

\param{generalized-boolean}---a \term{generalized boolean}.

\label Description::

%% 15.2.0 3
Returns \term{true}  if \param{list} is the \term{empty list}.
Returns \term{false} if \param{list} is a \term{cons}.

\label Examples::

\code
 (endp nil) \EV \term{true}
 (endp '(1 2)) \EV \term{false}
 (endp (cddr '(1 2))) \EV \term{true}
\endcode
                              
\label Side Effects:\None.

\label Affected By:\None.

\label Exceptional Situations::

\Shouldchecktype{list}{a \term{list}}

\label See Also:\None.

\label Notes::

The purpose of \funref{endp} is to test for the end of \param{proper list}.
Since \funref{endp} does not descend into a \term{cons}, 
it is well-defined to pass it a \term{dotted list}.
However, if shorter ``lists'' are iteratively produced 
by calling \funref{cdr} on such a \term{dotted list} 
and those ``lists'' are tested with \funref{endp},
a situation that has undefined consequences will eventually result 
when the \term{non-nil} \term{atom} (which is not in fact a \term{list})
finally becomes the argument to \funref{endp}.
Since this is the usual way in which \funref{endp} is used,
it is conservative programming style 
and consistent with the intent of \funref{endp}
to treat \funref{endp} as simply a function on \term{proper lists} 
which happens not to enforce an argument type of \term{proper list} except
when the argument is \term{atomic}.

\endcom

%%% ========== NULL
\begincom{null}\ftype{Function}

\issue{NOT-AND-NULL-RETURN-VALUE:X3J13-MAR-93}

\label Syntax::

\DefunWithValues null {object} {boolean}

\label Arguments and Values::

\param{object}---an \term{object}.

\param{boolean}---a \term{boolean}.

\label Description::

%% 6.2.2 3
\StrictPredicate{object}{the \term{empty list}}

\label Examples::   

\code
 (null '()) \EV T
 (null nil) \EV T
 (null t) \EV NIL
 (null 1) \EV NIL
\endcode

\endissue{NOT-AND-NULL-RETURN-VALUE:X3J13-MAR-93}

\label Side Effects:\None.

\label Affected By:\None.

\label Exceptional Situations:\None!

\label See Also::

\funref{not}

\label Notes::

%% 6.4.0 4
\funref{null} is intended to be used to test for the \term{empty list}
whereas \funref{not} is intended to be used to invert a \term{boolean}
(or \term{generalized boolean}).
Operationally, \funref{null} and \funref{not} compute the same result;
which to use is a matter of style.

\code
 (null \param{object}) \EQ (typep \param{object} 'null) \EQ (eq \param{object} '\empty)
\endcode

\endcom

%%% ========== NCONC
\begincom{nconc}\ftype{Function}

\label Syntax::

\DefunWithValues nconc {{\rest} lists} {concatenated-list}

\label Arguments and Values::

\issue{DOTTED-LIST-ARGUMENTS:CLARIFY}
\param{list}---each but the last must be a \term{list} 
	(which might be a \param{dotted list} but must not be a \term{circular list});
        the last \param{list} may be any \term{object}.
\endissue{DOTTED-LIST-ARGUMENTS:CLARIFY}

\param{concatenated-list}---a \term{list}.

\label Description::

% Note: Issue APPEND-DOTTED was repealed.

%% 15.2.0 28
Returns a \term{list} that is the concatenation of \param{lists}.
If no \param{lists} are supplied, \f{(nconc)} returns \nil.
\issue{REMF-DESTRUCTION-UNSPECIFIED:X3J13-MAR-89}
\funref{nconc} is defined using the following recursive relationship:
 
\code
 (nconc) \EV ()
 (nconc nil . \param{lists}) \EQ (nconc . \param{lists})
 (nconc \param{list}) \EV \param{list}
 (nconc \param{list-1} \param{list-2}) \EQ (progn (rplacd (last \param{list-1}) \param{list-2}) \param{list-1})
 (nconc \param{list-1} \param{list-2} . \param{lists}) \EQ (nconc (nconc \param{list-1} \param{list-2}) . \param{lists})
\endcode
\endissue{REMF-DESTRUCTION-UNSPECIFIED:X3J13-MAR-89}


\label Examples::

\code
 (nconc) \EV NIL
 (setq x '(a b c)) \EV (A B C)
 (setq y '(d e f)) \EV (D E F)
 (nconc x y) \EV (A B C D E F)
 x \EV (A B C D E F)
\endcode
Note, in the example, that the value of \f{x} is now different,
since its last \term{cons} 
has been \funref{rplacd}'d to the value of \f{y}.
If \f{(nconc x y)} were evaluated again,
it would yield a piece of a \term{circular list},
whose printed representation would be
\f{(A B C D E F D E F D E F ...)}, repeating forever;
if the \varref{*print-circle*} switch were \term{non-nil},
it would be printed as \f{(A B C . \#1=(D E F . \#1\#))}.

\issue{REMF-DESTRUCTION-UNSPECIFIED:X3J13-MAR-89}
\code
 (setq foo (list 'a 'b 'c 'd 'e)
       bar (list 'f 'g 'h 'i 'j)
       baz (list 'k 'l 'm)) \EV (K L M)
 (setq foo (nconc foo bar baz)) \EV (A B C D E F G H I J K L M)
 foo \EV (A B C D E F G H I J K L M)
 bar \EV (F G H I J K L M)
 baz \EV (K L M)

 (setq foo (list 'a 'b 'c 'd 'e)
       bar (list 'f 'g 'h 'i 'j)
       baz (list 'k 'l 'm)) \EV (K L M)
 (setq foo (nconc nil foo bar nil baz)) \EV (A B C D E F G H I J K L M) 
 foo \EV (A B C D E F G H I J K L M)
 bar \EV (F G H I J K L M)
 baz \EV (K L M)
\endcode

\endissue{REMF-DESTRUCTION-UNSPECIFIED:X3J13-MAR-89}

\label Side Effects::

The \param{lists} are modified rather than copied.

\label Affected By:\None.

\label Exceptional Situations:\None?

\label See Also::

\funref{append}, \funref{concatenate}

\label Notes:\None.

\endcom

%%% ========== APPEND
\begincom{append}\ftype{Function}

\label Syntax::

\DefunWithValues append {{\rest} lists} {result}

\label Arguments and Values:: 

\issue{DOTTED-LIST-ARGUMENTS:CLARIFY}
\param{list}---each must be a \term{proper list} except the last,
	       which may be any \term{object}.
\endissue{DOTTED-LIST-ARGUMENTS:CLARIFY}

\param{result}---an \term{object}.  This will be a \term{list}
		 unless the last \param{list} was not a \term{list}
%Sandra noted a but in the preceding.
        	    and all preceding \param{lists} were \term{null}.


\label Description::

%% Issue APPEND-DOTTED had an effect here, but was later rescinded.

%% 15.2.0 20
\funref{append} returns a new \param{list} that is the concatenation of
the copies.  \param{lists} are left unchanged; the \term{list structure} 
of each of \param{lists} except the last is copied.
%% 15.2.0 21
The last argument is not copied; it becomes the \term{cdr} of the 
final \term{dotted pair} of the concatenation of the preceding \param{lists},
or is returned directly if there are no preceding
% "non-empty" added with encouragement of Sandra and KAB. -kmp 28-Jan-92
\term{non-empty} 
\param{lists}.

\label Examples::

\code
 (append '(a b c) '(d e f) '() '(g)) \EV (A B C D E F G)
 (append '(a b c) 'd) \EV (A B C . D)
 (setq lst '(a b c)) \EV (A B C)
 (append lst '(d)) \EV (A B C D)
 lst \EV (A B C)
 (append) \EV NIL
 (append 'a) \EV A
\endcode
 
\label Affected By:\None.

\label Exceptional Situations:\None.

\label See Also::

\funref{nconc}, \funref{concatenate}

\label Notes:\None.

\endcom

%% Entries for REVAPPEND and NCONC consolidated. -kmp 29-Aug-93

%%% ========== REVAPPEND
%%% ========== NRECONC

\begincom{revappend, nreconc}\ftype{Function}

\issue{DOTTED-LIST-ARGUMENTS:CLARIFY}

\label Syntax::

\DefunWithValues revappend {list tail} {result-list}
\DefunWithValues nreconc   {list tail} {result-list}

\label Arguments and Values::

\param{list}---a \term{proper list}.

\param{tail}---an \term{object}. 

\param{result-list}---an \term{object}.

\label Description::

\funref{revappend} constructs a \term{copy}\meaning{2} of \param{list},
but with the \term{elements} in reverse order.  It then appends (as if
by \funref{nconc}) the \param{tail} to that reversed list and returns the result.

\funref{nreconc} reverses the order of \term{elements} in \param{list}
(as if by \funref{nreverse}).  It then appends (as if by \funref{nconc})
the \param{tail} to that reversed list and returns the result.

The resulting \term{list} shares \term{list structure} with \param{tail}.

\label Examples::

\code
 (let ((list-1 (list 1 2 3))
       (list-2 (list 'a 'b 'c)))
   (print (revappend list-1 list-2))
   (print (equal list-1 '(1 2 3)))
   (print (equal list-2 '(a b c))))
\OUT (3 2 1 A B C) 
\OUT T
\OUT T
\EV T

 (revappend '(1 2 3) '()) \EV (3 2 1)
 (revappend '(1 2 3) '(a . b)) \EV (3 2 1 A . B)
 (revappend '() '(a b c)) \EV (A B C)
 (revappend '(1 2 3) 'a) \EV (3 2 1 . A)
 (revappend '() 'a) \EV A   ;degenerate case

 (let ((list-1 '(1 2 3))
       (list-2 '(a b c)))
   (print (nreconc list-1 list-2))
   (print (equal list-1 '(1 2 3)))
   (print (equal list-2 '(a b c))))
\OUT (3 2 1 A B C) 
\OUT NIL
\OUT T
\EV T

\endcode

\label Side Effects::

\funref{revappend} does not modify either of its \term{arguments}.
\funref{nreconc} is permitted to modify \param{list} but not \param{tail}.

\issue{REMF-DESTRUCTION-UNSPECIFIED:X3J13-MAR-89}
Although it might be implemented differently,
\funref{nreconc} is constrained to have side-effect behavior equivalent to:

\code
 (nconc (nreverse \param{list}) \param{tail})
\endcode
\endissue{REMF-DESTRUCTION-UNSPECIFIED:X3J13-MAR-89}

\label Affected By:\None.

\label Exceptional Situations:\None.

\label See Also::

\funref{reverse},
\funref{nreverse},
\funref{nconc}

\label Notes::

%% 15.2.0 27
%% 15.2.0 30

The following functional equivalences are true, 
although good \term{implementations} will typically use a faster algorithm for
achieving the same effect:

\code
 (revappend \param{list} \param{tail}) \EQ (nconc (reverse \param{list}) \param{tail})
 (nreconc \param{list} \param{tail}) \EQ (nconc (nreverse \param{list}) \param{tail})
\endcode

\endissue{DOTTED-LIST-ARGUMENTS:CLARIFY}

\endcom



%% Entries for REVAPPEND and NCONC consolidated. -kmp 29-Aug-93
% %%% ========== NRECONC
% \begincom{nreconc}\ftype{Function}
% 
% \label Syntax::
% 
% \DefunWithValues nreconc {list-1 list-2} {result-list}
% 
% \label Arguments and Values:: 
% 
% \issue{DOTTED-LIST-ARGUMENTS:CLARIFY}
% 
% \param{list-1}---a \term{proper list}.
% 
% \param{list-2}---an \term{object}.
% 
% \endissue{DOTTED-LIST-ARGUMENTS:CLARIFY}
% 
% \param{result-list}---an \term{object}. 
% 
% \label Description::
% 
% \funref{nreconc} reverses the order of the elements in 
% \param{list-1} and appends \param{list-2} to \param{list-1} 
% \issue{DOTTED-LIST-ARGUMENTS:CLARIFY}
% (as if by \funref{nconc}).
% \endissue{DOTTED-LIST-ARGUMENTS:CLARIFY}
% The new \param{list-1} is returned.
% 
% \label Examples::
% 
% \code
%  (defparameter *list-1* (list 1 2 3))
%  (defparameter *list-2* (list 'a 'b 'c))
%  (nreconc *list-1* *list-2*) \EV (3 2 1 A B C)
%  *list-1* \EV \term{implementation-dependent}
%  *list-2* \EV (A B C)
% 
%  (nreconc (list) 'a) \EV A  ;degenerate situation
% \endcode
% 
% \issue{DOTTED-LIST-ARGUMENTS:CLARIFY}
% % \code
% %  (nreconc (cons 1 2) nil) \EV (1)
% % \endcode
% \endissue{DOTTED-LIST-ARGUMENTS:CLARIFY}
% 
% \label Side Effects::
% 
% \param{list-1} is modified.
% 
% \issue{REMF-DESTRUCTION-UNSPECIFIED:X3J13-MAR-89}
% \funref{nreconc} is constrained to have side-effect behavior equivalent to:
% 
% \f{(nconc (nreverse \param{list-1}) \param{list-2})}.
% \endissue{REMF-DESTRUCTION-UNSPECIFIED:X3J13-MAR-89}
% 
% \label Affected By:\None.
% 
% \label Exceptional Situations:\None?
% 
% \label See Also::
% 
% \funref{revappend}
% 
% \label Notes::
% %% 15.2.0 30
% \f{(nreconc x y)} is exactly the same as 
% \f{(nconc (nreverse x) y)} except that it is potentially more efficient.
% 
% \endcom

%% Entries for REVAPPEND and NCONC consolidated. -kmp 29-Aug-93
% %%% ========== REVAPPEND
% \begincom{revappend}\ftype{Function}
% 
% \label Syntax::
% 
% \DefunWithValues revappend {list-1 list-2} {result-list}
% 
% \label Arguments and Values::
% 
% \issue{DOTTED-LIST-ARGUMENTS:CLARIFY}
% \param{list-1}---a \term{proper list}.
% 
% \param{list-2}---an \term{object}. 
% 
% \param{result-list}---an \term{object}.
% \endissue{DOTTED-LIST-ARGUMENTS:CLARIFY}
% 
% \label Description::
% 
% \issue{DOTTED-LIST-ARGUMENTS:CLARIFY}
% Constructs a \term{fresh} \term{list} that is a \term{copy} of \param{list-1}
% but with the \term{elements} in reverse order,
% and then appends \param{list-2} to that \term{list}
% (as if by \funref{nconc}).
% %(as if by \funref{append}).
% \endissue{DOTTED-LIST-ARGUMENTS:CLARIFY}
% 
% The resulting \term{list} shares \term{list structure} with \param{list-2}.
% 
% \label Examples::
% 
% \issue{DOTTED-LIST-ARGUMENTS:CLARIFY}
% \code
%  (setq lst1 '(1 2 3)
%         lst2 '(a b c))  \EV (A B C)
%  (revappend lst1 lst2) \EV (3 2 1 A B C)
%  lst1 \EV (1 2 3)
%  lst2 \EV (A B C)
%  (revappend '(1 2 3) '(a . b)) \EV (3 2 1 A . B)
%  (revappend nil '(a b c)) \EV (A B C)
%  (revappend '() 'a) \EV A ;degenerate case
% \endcode
% 
% % \code
% %  (revappend '(1 . 2) '(a b c)) \EV (1 A B C)
% % \endcode
% \endissue{DOTTED-LIST-ARGUMENTS:CLARIFY}
% 
% \label Side Effects:\None.
% 
% \label Affected By:\None.
% 
% \label Exceptional Situations:\None.
% 
% \label See Also::
% 
% \funref{nreconc}
% 
% \label Notes::
% %% 15.2.0 27
% 
% %"reverse" => "nconc" per Barrett
% \code
%  (revappend x y) \EQ (nconc (reverse x) y)
% \endcode
% 
% \endcom

%%% ========== NBUTLAST
%%% ========== BUTLAST
\begincom{butlast, nbutlast}\ftype{Function}

\label Syntax::

\DefunWithValues butlast  {list {\opt} n} {result-list}
\DefunWithValues nbutlast {list {\opt} n} {result-list}

\label Arguments and Values::

\param{list}---a \term{list}, 
\issue{DOTTED-LIST-ARGUMENTS:CLARIFY}
	       which might be a \term{dotted list} but must not be a \term{circular list}.
\endissue{DOTTED-LIST-ARGUMENTS:CLARIFY}

\issue{BUTLAST-NEGATIVE:SHOULD-SIGNAL}
\param{n}---a non-negative \term{integer}.
\endissue{BUTLAST-NEGATIVE:SHOULD-SIGNAL}

\param{result-list}---a \term{list}.

\label Description::

%% 15.2.0 36
\funref{butlast} returns a copy of \param{list} from which the last
\param{n}
\issue{DOTTED-LIST-ARGUMENTS:CLARIFY}
%elements
conses
\endissue{DOTTED-LIST-ARGUMENTS:CLARIFY}
have been omitted.
If \param{n} is not supplied, its value is 1.  
If there are fewer than \param{n} 
\issue{DOTTED-LIST-ARGUMENTS:CLARIFY}
%elements
conses
\endissue{DOTTED-LIST-ARGUMENTS:CLARIFY}
in \param{list},
\nil\ is returned and, in the case of \funref{nbutlast},
\param{list} is not modified.  
%Barmar notes that if there are n elements, NIL is returned, too.
%He suggests says "fewer than n+1".
%But KMP thinks the definition already covers that, and the reason for the
%exception about NIL is to clarify the behavior in the cases not covered.

%% 15.2.0 37
\funref{nbutlast} is like \funref{butlast}, but \funref{nbutlast} 
may modify \param{list}.
It changes the \term{cdr} of
the \term{cons} \param{n}+1 from the end of the \param{list} to \nil.  

\label Examples::
\code
 (setq lst '(1 2 3 4 5 6 7 8 9)) \EV (1 2 3 4 5 6 7 8 9)
 (butlast lst) \EV (1 2 3 4 5 6 7 8)
 (butlast lst 5) \EV (1 2 3 4)
 (butlast lst (+ 5 5)) \EV NIL
 lst \EV (1 2 3 4 5 6 7 8 9)
 (nbutlast lst 3) \EV (1 2 3 4 5 6)
 lst \EV (1 2 3 4 5 6)
 (nbutlast lst 99) \EV NIL
 lst \EV (1 2 3 4 5 6)
 (butlast '(a b c d)) \EV (A B C)
 (butlast '((a b) (c d))) \EV ((A B))
 (butlast '(a)) \EV NIL
 (butlast nil) \EV NIL
 (setq foo (list 'a 'b 'c 'd)) \EV (A B C D)
 (nbutlast foo) \EV (A B C)
 foo \EV (A B C)
 (nbutlast (list 'a)) \EV NIL
 (nbutlast '()) \EV NIL
\endcode

% The following example was present but no one could figure out what
% made anyone think this would reliably work, so I removed it. -kmp 14-Feb-92
%
% (butlast '(1 2 . 3)) \EV (1)

\label Affected By:\None.

\label Exceptional Situations::

\issue{DOTTED-LIST-ARGUMENTS:CLARIFY}
\Shouldchecktype{list}{a \term{proper list} or a \term{dotted list}}
\endissue{DOTTED-LIST-ARGUMENTS:CLARIFY}
\issue{BUTLAST-NEGATIVE:SHOULD-SIGNAL}
\Shouldchecktype{n}{a non-negative \term{integer}}
\endissue{BUTLAST-NEGATIVE:SHOULD-SIGNAL}

\label See Also:\None.

\label Notes::

\issue{DOTTED-LIST-ARGUMENTS:CLARIFY}
\code
 (butlast \param{list} \param{n}) \EQ (ldiff \param{list} (last \param{list} \param{n}))
\endcode
\endissue{DOTTED-LIST-ARGUMENTS:CLARIFY}

\endcom      

%%% ========== LAST
\begincom{last}\ftype{Function}      

\label Syntax::

\DefunWithValues last {list {\opt} n} {tail}

\label Arguments and Values::

\param{list}---a \term{list},
\issue{DOTTED-LIST-ARGUMENTS:CLARIFY}
               which might be a \term{dotted list} but must not be a \term{circular list}.
\endissue{DOTTED-LIST-ARGUMENTS:CLARIFY}

\issue{LAST-N}
\param{n}---a non-negative \term{integer}.
 \Default{\f{1}}
\endissue{LAST-N}

\param{tail}---an \term{object}.

\label Description::

\issue{LAST-N}
%% 15.2.0 16
\funref{last} returns the last \param{n} \term{conses} 
 (not the last \param{n} elements) of \param{list}).
If \param{list} is \empty, \funref{last} returns \empty.

If \param{n} is zero,
  the atom that terminates \param{list} is returned.
If \param{n} is greater than or equal to the number of \term{cons} cells in \param{list},
  the result is \param{list}.
\endissue{LAST-N}

\label Examples::

\issue{LAST-N}
\code
 (last nil) \EV NIL
 (last '(1 2 3)) \EV (3)
 (last '(1 2 . 3)) \EV (2 . 3)
 (setq x (list 'a 'b 'c 'd)) \EV (A B C D)
 (last x) \EV (D)
 (rplacd (last x) (list 'e 'f)) x \EV (A B C D E F)
 (last x) \EV (F)

 (last '(a b c))   \EV (C)

 (last '(a b c) 0) \EV ()
 (last '(a b c) 1) \EV (C)
 (last '(a b c) 2) \EV (B C)
 (last '(a b c) 3) \EV (A B C)
 (last '(a b c) 4) \EV (A B C)

 (last '(a . b) 0) \EV B
 (last '(a . b) 1) \EV (A . B)
 (last '(a . b) 2) \EV (A . B)
\endcode
\endissue{LAST-N}

\label Side Effects:\None.

\label Affected By:\None.

\label Exceptional Situations::

\issue{LAST-N}
The consequences are undefined if \param{list} is a \term{circular list}.
\endissue{LAST-N}
\Shouldchecktype{n}{a non-negative \term{integer}}

\label See Also::

\funref{butlast},
\funref{nth}

\label Notes::                                             

\issue{LAST-N}
The following code could be used to define \funref{last}.

\code
 (defun last (list &optional (n 1))
   (check-type n (integer 0))
   (do ((l list (cdr l))
        (r list)
        (i 0 (+ i 1)))
       ((atom l) r)
     (if (>= i n) (pop r))))
\endcode
\endissue{LAST-N}
\endcom

%% Tentatively merged entries for LDIFF and TAILP. -kmp 29-Aug-93

%%% ========== LDIFF
%%% ========== TAILP
\begincom{ldiff, tailp}\ftype{Function}

\label Syntax::

\DefunWithValues ldiff {list object} {result-list}
\DefunWithValues tailp {object list} {generalized-boolean}

\label Arguments and Values::

\param{list}---a \term{list},
\issue{DOTTED-LIST-ARGUMENTS:CLARIFY}
                   which might be a \term{dotted list}.

\param{object}---an \term{object}.
\endissue{DOTTED-LIST-ARGUMENTS:CLARIFY}

\param{result-list}---a \term{list}.

%% Per X3J13. -kmp 05-Oct-93
\param{generalized-boolean}---a \term{generalized boolean}.

\label Description::

\issue{TAILP-NIL:T}
%% 15.5.0 7
%% (CLtL definition superseded by later cleanup issue clarifying behavior.)
If \param{object} is the \term{same} as some \term{tail} of \param{list},
\funref{tailp} returns \term{true};
otherwise, it returns \term{false}.
\endissue{TAILP-NIL:T}

%% 15.2.0 38
If \param{object} is the \term{same} as some \term{tail} of \param{list},
\funref{ldiff} returns a \term{fresh} \term{list} 
of the \term{elements} of \term{list} 
that precede \funref{object} in the \term{list structure} of \param{list};
otherwise, it returns a \term{copy}\meaning{2} of \param{list}.

\label Examples::

\issue{TAILP-NIL:T}
\code
 (let ((lists '#((a b c) (a b c . d))))
   (dotimes (i (length lists)) ()
     (let ((list (aref lists i)))
       (format t "~2&list=~S ~21T(tailp object list)~
                  ~44T(ldiff list object)~%" list)
         (let ((objects (vector list (cddr list) (copy-list (cddr list))
                                '(f g h) '() 'd 'x)))
           (dotimes (j (length objects)) ()
             (let ((object (aref objects j)))
               (format t "~& object=~S ~21T~S ~44T~S"
                       object (tailp object list) (ldiff list object))))))))
\OUT 
\OUT list=(A B C)         (tailp object list)    (ldiff list object)
\OUT  object=(A B C)      T                      NIL
\OUT  object=(C)          T                      (A B)
\OUT  object=(C)          NIL                    (A B C)
\OUT  object=(F G H)      NIL                    (A B C)
\OUT  object=NIL          T                      (A B C)
\OUT  object=D            NIL                    (A B C)
\OUT  object=X            NIL                    (A B C)
\OUT 
\OUT list=(A B C . D)     (tailp object list)    (ldiff list object)
\OUT  object=(A B C . D)  T                      NIL
\OUT  object=(C . D)      T                      (A B)
\OUT  object=(C . D)      NIL                    (A B C . D)
\OUT  object=(F G H)      NIL                    (A B C . D)
\OUT  object=NIL          NIL                    (A B C . D)
\OUT  object=D            T                      (A B C)
\OUT  object=X            NIL                    (A B C . D)
\EV NIL
\endcode
\endissue{TAILP-NIL:T}

\label Side Effects::

Neither \funref{ldiff} nor \funref{tailp} modifies either of its \term{arguments}.

\label Affected By:\None.

\label Exceptional Situations::

\issue{DOTTED-LIST-ARGUMENTS:CLARIFY}
\Lazychecktype{list}{a \term{proper list} or a \term{dotted list}}
\endissue{DOTTED-LIST-ARGUMENTS:CLARIFY}

\label See Also::

\funref{set-difference}

\label Notes::

If the \param{list} is a \term{circular list},
\funref{tailp} will reliably \term{yield} a \term{value} 
only if the given \param{object} is in fact a \term{tail} of \param{list}.
Otherwise, the consequences are unspecified:
a given \term{implementation} which detects the circularity must return \term{false},
but since an \term{implementation} is not obliged to detect such a \term{situation},
\funref{tailp} might just loop indefinitely without returning in that case.

\issue{TAILP-NIL:T}

\funref{tailp} could be defined as follows:

\code
 (defun tailp (object list)
   (do ((list list (cdr list)))
       ((atom list) (eql list object))
      (if (eql object list)
          (return t))))
\endcode

and \funref{ldiff} could be defined by:

%% !!! I just up the following based on the Description.
%%     Does everyone agree it's right? -kmp 29-Aug-93
\code
(defun ldiff (list object)
  (do ((list list (cdr list))
       (r '() (cons (car list) r)))
      ((atom list)
       (if (eql list object) (nreverse r) (nreconc r list)))
    (when (eql object list)
      (return (nreverse r)))))
\endcode

\issue{DOTTED-LIST-ARGUMENTS:CLARIFY}
% This stuff probably isn't needed. -kmp 7-Jan-91
%
% Since the \param{list} can be a \term{dotted list},
% the end test must be \funref{atom}, not \funref{endp}.
% For example, if \f{(tailp \i{x} l)} returns \term{true},
% it means that there is an \i{n} such that \f{(nthcdr \i{n} \param{list})} returns \i{x}.
%
% Note that it doesn't follow that if \term{tailp} returns \nil,
% it is safe to go \funref{nthcdr}'s into the \param{list} looking
% for \i{x}, since the \param{list} might be a \term{dotted list}
% and \funref{nthcdr} might hit bad data.
\endissue{DOTTED-LIST-ARGUMENTS:CLARIFY}

\endissue{TAILP-NIL:T}

\endcom

% %%% ========== LDIFF
% \begincom{ldiff}\ftype{Function}
% 
% \label Syntax::
% 
% \DefunWithValues ldiff {list object} {result-list}
% 
% \label Arguments and Values::
% 
% \param{list}---a \term{list},
% \issue{DOTTED-LIST-ARGUMENTS:CLARIFY}
%                    which might be a \term{dotted list}.
% 
% \param{object}---an \term{object}.
% \endissue{DOTTED-LIST-ARGUMENTS:CLARIFY}
% 
% \param{result-list}---a \term{list}.
% 
% \label Description::
% 
% %!!! Rewrite in terms of glossary term "sublist" if we can get the issue
% %    of dottedness resolved. -kmp 27-May-91
% 
% %% 15.2.0 38
% Returns a \term{fresh} \term{list} whose \term{elements} are 
% those \term{elements} of \param{list} that appear before \param{object}.
% If \param{object} is not a tail of \param{list} or
% if \param{object} is \nil,
% then a copy of the entire \param{list} is returned.
% \param{list} is not destroyed.
% 
% % KMP: Can the list be dotted?
% %      What is the status of (LDIFF '(A B C . 3) 3)?
% % Barrett: The description seems to imply that it cannot be dotted.
% 
% \label Examples::
% 
% \code
%  (setq x '(a b c d e)) \EV (A B C D E)
%  (setq y (cdddr x)) \EV (D E)
%  (ldiff x y) \EV (A B C)
%  (ldiff x (copy-list y)) \EV (A B C D E)
% \endcode
% 
% \label Side Effects:\None.
% 
% \label Affected By:\None.
% 
% \label Exceptional Situations::
% 
% \issue{DOTTED-LIST-ARGUMENTS:CLARIFY}
% \Lazychecktype{list}{a \term{proper list} or a \term{dotted list}}
% \endissue{DOTTED-LIST-ARGUMENTS:CLARIFY}
% 
% \label See Also::
% 
% \funref{tailp},
% \funref{set-difference}
% 
% \label Notes:\None.
% 
% \endcom
% 
% %%% ========== TAILP
% \begincom{tailp}\ftype{Function}
% 
% \label Syntax::
% 
% \DefunWithValues tailp {object list} {generalized-boolean}
% 
% \label Arguments and Values:: 
% 
% \param{object}---an \term{object}.
% 
% \param{list}---a \term{list},
% \issue{DOTTED-LIST-ARGUMENTS:CLARIFY}
%                    which might be a \term{dotted list}.
% \endissue{DOTTED-LIST-ARGUMENTS:CLARIFY}
% 
% \param{generalized-boolean}---a \term{generalized boolean}.
% 
% \label Description::
% 
% \issue{TAILP-NIL:T}
% %% 15.5.0 7
% %% (CLtL definition superseded by later cleanup issue clarifying behavior.)
% 
% Returns \term{true} if and only if \param{object} is a \term{tail} of \param{list};
% otherwise returns \term{false}.  The predicate used for comparison is \funref{eql}.
% 
% Since the \param{list} can be a \term{dotted list},
% the end test used by \funref{tailp} must be \funref{atom}, not \funref{endp}.
% That is, if \f{(tailp \i{x} l)} returns \term{true},
% it means that there is an \i{n} such that \f{(nthcdr \i{n} \param{list})} returns \i{x}.
%  
% \label Examples::
% 
% \code
%  (let ((x '(b c))) (tailp x (cons 'a x))) \EV \term{true}
%  (tailp '(x y) '(a b c)) \EV \term{false}
%  (tailp '() '(a b c)) \EV \term{true}
%  (tailp 3 '(a b c)) \EV \term{false}
%  (tailp 3 '(a b c . 3)) \EV \term{true}
%  (tailp '(x y) '(a b c . 3)) \EV \term{false}
% \endcode
% 
% \endissue{TAILP-NIL:T}
% 
% \label Side Effects:\None.
% 
% \label Affected By:\None.
% 
% \label Exceptional Situations::
% 
% \issue{DOTTED-LIST-ARGUMENTS:CLARIFY}
% \Lazychecktype{list}{a \term{proper list} or a \term{dotted list}}
% \endissue{DOTTED-LIST-ARGUMENTS:CLARIFY}
% 
% \label See Also::
% 
% \funref{ldiff}
% 
% \label Notes::
% 
% If the \param{list} is a \term{circular list},
% \funref{tailp} will reliably \term{yield} a \term{value} 
% only if the given \param{object} is in fact a \term{subtail}.
% Otherwise, the consequences are unspecified:
% a given \term{implementation} which detects the circularity must return \term{false},
% but since an \term{implementation} is not obliged to detect such a \term{situation},
% \funref{tailp} might just loop indefinitely without returning in that case.
% 
% \issue{TAILP-NIL:T}
% 
% \funref{tailp} could be defined as follows:
% 
% \code
%  (defun tailp (sublist list)
%    (do ((list list (cdr list)))
%        ((atom list) (eql list sublist))
%       (if (eql sublist list)
%           (return t))))
% \endcode
% 
% % This probably isn't needed. -kmp 7-Jan-91
% %
% % Note that it doesn't follow that if \term{tailp} returns \nil,
% % it is safe to go \funref{nthcdr}'s into the \param{list} looking
% % for \i{x}, since the \param{list} might be a \term{dotted list}
% % and \funref{nthcdr} might hit bad data.
% 
% \endissue{TAILP-NIL:T}
% 
% \endcom

%%% ========== NTHCDR
\begincom{nthcdr}\ftype{Function}

\label Syntax::

\DefunWithValues nthcdr {n list} {tail}

\label Arguments and Values::

\issue{ARGUMENTS-UNDERSPECIFIED:SPECIFY}
\param{n}---a non-negative \term{integer}.
\endissue{ARGUMENTS-UNDERSPECIFIED:SPECIFY}

\param{list}---a \term{list},
\issue{DOTTED-LIST-ARGUMENTS:CLARIFY}
	       which might be a \term{dotted list} or a \term{circular list}.
\endissue{DOTTED-LIST-ARGUMENTS:CLARIFY}

\param{tail}---an \term{object}.
%\term{tail} of the \param{list}.

\label Description::

%% 15.2.0 14
%% Described using more descriptive terminology, 
%% in a way that doesn't get confused about whether n is 0-based or 1-based. -kmp 29-Aug-93
%Returns the \param{n}th successive \term{cdr} of \param{list}.
Returns the \term{tail} of \param{list} that would be obtained by calling \funref{cdr}
\param{n} times in succession.

\label Examples::

% I added an example of high-safety error, to demonstrate relation of NTHCDR and CDR.
% -kmp 26-Aug-93
\code
 (nthcdr 0 '()) \EV NIL
 (nthcdr 3 '()) \EV NIL
 (nthcdr 0 '(a b c)) \EV (A B C)
 (nthcdr 2 '(a b c)) \EV (C)
 (nthcdr 4 '(a b c)) \EV ()
 (nthcdr 1 '(0 . 1)) \EV 1

 (locally (declare (optimize (safety 3)))
   (nthcdr 3 '(0 . 1)))
 Error: Attempted to take CDR of 1.
\endcode

\label Side Effects:\None.

\label Affected By:\None.

\label Exceptional Situations::

%% I added this stuff because it seemed consistent with the other entries. -kmp 26-Aug-93

\Shouldchecktype{n}{a non-negative \term{integer}}

For \param{n} being an integer greater than \f{1},
the error checking done by \f{(nthcdr \param{n} \param{list})}
is the same as for \f{(nthcdr (- \param{n} 1) (cdr \param{list}))};
\seefun{cdr}.

\label See Also::

\funref{cdr},
\funref{nth},
\funref{rest}

\label Notes:\None.

\endcom

%%% ========== REST
\begincom{rest}\ftype{Accessor}

\label Syntax::

\DefunWithValues rest {list} {tail}
\Defsetf         rest {list} {new-tail}

\label Arguments and Values::

\param{list}---a \term{list},
\issue{DOTTED-LIST-ARGUMENTS:CLARIFY}
	       which might be a \term{dotted list} or a \term{circular list}.
\endissue{DOTTED-LIST-ARGUMENTS:CLARIFY}

\param{tail}---an \term{object}.

\label Description::

%% 15.2.0 13
\funref{rest} performs the same operation as \funref{cdr},
but mnemonically complements \funref{first}.
Specifically,

\code
 (rest \param{list}) \EQ (cdr \param{list})
 (setf (rest \param{list}) \param{new-tail}) \EQ (setf (cdr \param{list}) \param{new-tail})
\endcode

% If \param{list} is a \term{cons},
% \funref{rest} returns the portion that follows the first \term{element}.
% If \param{list} is the \term{empty list},
% \funref{rest} returns the \term{empty list}.
% 
% \macref{setf} may be used with \funref{rest} to change 
% the \term{cdr} of a \term{non-empty} \term{list} (\ie a \term{cons}).

\label Examples::

\code
 (rest '(1 2)) \EV (2)
 (rest '(1 . 2)) \EV 2
 (rest '(1)) \EV NIL
 (setq *cons* '(1 . 2)) \EV (1 . 2)
 (setf (rest *cons*) "two") \EV "two"
 *cons* \EV (1 . "two")
\endcode

\label Side Effects:\None.

\label Affected By:\None.

\label Exceptional Situations:\None.

\label See Also::

\funref{cdr},
\funref{nthcdr}

\label Notes::

\funref{rest} is often preferred stylistically over \funref{cdr}
when the argument is to being subjectively viewed as a \term{list} 
rather than as a \term{cons}.

\endcom

%%% ========== MEMBER
%%% ========== MEMBER-IF
%%% ========== MEMBER-IF-NOT
\begincom{member, member-if, member-if-not}\ftype{Function}

\label Syntax::

\DefunWithValues member        {item      list {\key} key test test-not} {tail}
\DefunWithValues member-if     {predicate list {\key} key} {tail}
\DefunWithValues member-if-not {predicate list {\key} key} {tail}

\label Arguments and Values:: 

\param{item}---an \term{object}.

\param{list}---a \term{proper list}.

\param{predicate}---a \term{designator} for 
		    a \term{function} of one \term{argument}
		    that returns a \term{generalized boolean}.

\param{test}---a \term{designator} for a \term{function} of two \term{arguments}
  that returns a \term{generalized boolean}.

\param{test-not}---a \term{designator} for 
  a \term{function} of two \term{arguments}
  that returns a \term{generalized boolean}.

\param{key}---a \term{designator} for a \term{function} of one argument,
  or \nil.

\param{tail}---a \term{list}.

\label Description::

%% 15.5.0 4
\funref{member}, \funref{member-if}, and \funref{member-if-not} each
search \param{list} for \param{item} or for a top-level element that 
\term{satisfies the test}.  The argument to the \param{predicate} function 
is an element of \param{list}.

%% Implied by "satifies the test". -kmp 15-Feb-92
% If \kwd{test} or \kwd{test-not} is not supplied, 
% \funref{eql} is used.
% The first argument to the \kwd{test} 
% or \kwd{test-not} function is \param{item}, and the second argument
% is an element of \param{list} as returned by the \kwd{key} function.
% It is an error to supply both \kwd{test} and 
% \kwd{test-not} 
% in the same function call.
% 
% If \kwd{key} is supplied, it is used to
% extract the part to be tested from the \param{list} element.
% The argument to the \kwd{key} function 
% is an element of \param{list}; typically, the  \kwd{key} function 
% returns part of the element of
% \param{list} it was given.
% If \kwd{key} is not supplied or \nil, the \param{list} 
% element is used.

If some element \term{satisfies the test},
the tail of \param{list} beginning
with this element is returned; otherwise \nil\ is returned.

\param{list} is searched on the top level only. 

\label Examples::

\code
 (member 2 '(1 2 3)) \EV (2 3)                                 
 (member 2 '((1 . 2) (3 . 4)) :test-not #'= :key #'cdr) \EV ((3 . 4))
 (member 'e '(a b c d)) \EV NIL
\endcode


%!!! I'm suspicious of this dotted list.
\code
 (member-if #'listp '(a b nil c d)) \EV (NIL C D)
 (member-if #'numberp '(a #\\Space 5/3 foo)) \EV (5/3 FOO)
 (member-if-not #'zerop 
                 '(3 6 9 11 . 12)
                 :key #'(lambda (x) (mod x 3))) \EV (11 . 12)
\endcode

\label Side Effects:\None.

\label Affected By:\None.

\label Exceptional Situations::

\Lazychecktype{list}{a \term{proper list}}

\label See Also::

\funref{find},
\funref{position},
\issue{MAPPING-DESTRUCTIVE-INTERACTION:EXPLICITLY-VAGUE}
{\secref\TraversalRules}
\endissue{MAPPING-DESTRUCTIVE-INTERACTION:EXPLICITLY-VAGUE}

\label Notes::

\issue{TEST-NOT-IF-NOT:FLUSH-ALL}
The \kwd{test-not} parameter is deprecated.
\endissue{TEST-NOT-IF-NOT:FLUSH-ALL}

\issue{TEST-NOT-IF-NOT:FLUSH-ALL}
\Thefunction{member-if-not} is deprecated.
\endissue{TEST-NOT-IF-NOT:FLUSH-ALL}

In the following

\code
 (member 'a '(g (a y) c a d e a f)) \EV (A D E A F)
\endcode

the value returned by \funref{member} is \term{identical} to the portion
of the \term{list} beginning with \f{a}.  Thus \funref{rplaca} on the
result of \funref{member} can be used to alter the part of the \term{list}
where \f{a} was found (assuming a check has been made that \funref{member}
did not return \nil).

\endcom

%-------------------- List Mapping --------------------


%%% ========== MAPCAR
%%% ========== MAPLIST
%%% ========== MAPC
%%% ========== MAPL
%%% ========== MAPCAN
%%% ========== MAPCON
\begincom{mapc, mapcar, mapcan, mapl, maplist, mapcon}\ftype{Function}

\label Syntax::

\DefunWithValues mapc    {function {\rest} \plus{lists}} {list-1}
\DefunWithValues mapcar  {function {\rest} \plus{lists}} {result-list}
\DefunWithValues mapcan  {function {\rest} \plus{lists}} {concatenated-results}
\DefunWithValues mapl    {function {\rest} \plus{lists}} {list-1}
\DefunWithValues maplist {function {\rest} \plus{lists}} {result-list}
\DefunWithValues mapcon  {function {\rest} \plus{lists}} {concatenated-results}

\label Arguments and Values:: 

\param{function}---a \term{designator} for a \term{function} 
  that must take as many \term{arguments} as there are \param{lists}.

\issue{DOTTED-LIST-ARGUMENTS:CLARIFY}
\param{list}---a \term{proper list}.

\param{list-1}---the first \param{list} (which must be a \term{proper list}).
\endissue{DOTTED-LIST-ARGUMENTS:CLARIFY}

\param{result-list}---a \term{list}.
                   
\param{concatenated-results}---a \term{list}.

\label Description::

The mapping operation involves applying \param{function} to
successive sets of arguments in which
one argument is obtained from each \term{sequence}.
Except for \funref{mapc} and \funref{mapl},
the result contains the results returned by \param{function}.
In the cases of \funref{mapc} and \funref{mapl},
the resulting \term{sequence} is \param{list}.

%% 14.2.0 6
\param{function} is called
first on all the elements with index \f{0}, then on all those
with index \f{1}, and so on.
\param{result-type} specifies the \term{type} of
the 
resulting \term{sequence}.
\issue{FUNCTION-TYPE:X3J13-MARCH-88}
If \param{function} is a \term{symbol}, it is \funref{coerce}d
to a \term{function} as if by \funref{symbol-function}.
\endissue{FUNCTION-TYPE:X3J13-MARCH-88}

%% 7.8.4 4
\funref{mapcar} operates on successive \term{elements} of the \param{lists}.
\param{function} is applied to the first \term{element} of each \param{list},
then to the second \term{element} of each \param{list}, and so on.
The iteration terminates when the shortest \param{list} runs out,
and excess elements in other lists are ignored.
The value returned by \funref{mapcar} is a \term{list}
of the results of successive calls to \param{function}.

%% 7.8.4 6             
\funref{mapc} is like \funref{mapcar} except that the results of 
applying \param{function} are not accumulated.
The \param{list} argument is returned.

%% 7.8.4 5
\funref{maplist} is like \funref{mapcar} except that
\param{function} is applied to successive sublists of the \param{lists}.
\param{function} 
is first applied to the \param{lists} themselves, 
and then to the \term{cdr} of each
\param{list}, and then to the \term{cdr} of the \term{cdr}
of each \param{list}, and so on.  

\funref{mapl} is like \funref{maplist} except that the results of 
applying \param{function} are not accumulated;
\param{list-1} is returned.
  
%% 7.8.4 8
\funref{mapcan} and \funref{mapcon} are like \funref{mapcar} and
\funref{maplist} respectively, except that the results of
applying \param{function} are combined 
into a \term{list} by the use of \funref{nconc}
rather than \funref{list}.
That is,

\code
 (mapcon f x1 ... xn)
   \EQ (apply #'nconc (maplist f x1 ... xn))
\endcode
and similarly for the relationship between \funref{mapcan} 
and \funref{mapcar}.

\label Examples::

\code
 (mapcar #'car '((1 a) (2 b) (3 c))) \EV (1 2 3) 
 (mapcar #'abs '(3 -4 2 -5 -6)) \EV (3 4 2 5 6)
 (mapcar #'cons '(a b c) '(1 2 3)) \EV ((A . 1) (B . 2) (C . 3))

 (maplist #'append '(1 2 3 4) '(1 2) '(1 2 3)) 
\EV ((1 2 3 4 1 2 1 2 3) (2 3 4 2 2 3)) 
 (maplist #'(lambda (x) (cons 'foo x)) '(a b c d))
\EV ((FOO A B C D) (FOO B C D) (FOO C D) (FOO D))
 (maplist #'(lambda (x) (if (member (car x) (cdr x)) 0 1)) '(a b a c d b c))
\EV (0 0 1 0 1 1 1)
;An entry is 1 if the corresponding element of the input
;  list was the last instance of that element in the input list.

 (setq dummy nil) \EV NIL 
 (mapc #'(lambda (&rest x) (setq dummy (append dummy x)))
        '(1 2 3 4)
        '(a b c d e)
        '(x y z)) \EV (1 2 3 4) 
 dummy \EV (1 A X 2 B Y 3 C Z)                   

 (setq dummy nil) \EV NIL 
 (mapl #'(lambda (x) (push x dummy)) '(1 2 3 4)) \EV (1 2 3 4) 
 dummy \EV ((4) (3 4) (2 3 4) (1 2 3 4)) 

 (mapcan #'(lambda (x y) (if (null x) nil (list x y)))
          '(nil nil nil d e)
          '(1 2 3 4 5 6)) \EV (D 4 E 5) 
 (mapcan #'(lambda (x) (and (numberp x) (list x)))
          '(a 1 b c 3 4 d 5))
\EV (1 3 4 5)
\endcode
In this case the function serves as a filter; 
this is a standard \Lisp\ idiom using \funref{mapcan}.

\code
 (mapcon #'list '(1 2 3 4)) \EV ((1 2 3 4) (2 3 4) (3 4) (4)) 
\endcode

\label Affected By:\None.

\label Exceptional Situations::

\Lazycheckanytype{list}{a \term{proper list}}

\label See Also::

\macref{dolist},
\funref{map},
\issue{MAPPING-DESTRUCTIVE-INTERACTION:EXPLICITLY-VAGUE}
{\secref\TraversalRules}
\endissue{MAPPING-DESTRUCTIVE-INTERACTION:EXPLICITLY-VAGUE}

\label Notes:\None.

\endcom

%-------------------- Alists --------------------

%%% ========== ACONS
\begincom{acons}\ftype{Function}

\label Syntax::

\DefunWithValues acons {key datum alist} {new-alist}

\label Arguments and Values:: 

\param{key}---an \term{object}.

\param{datum}---an \term{object}.

\param{alist}---an \term{association list}.

\param{new-alist}---an \term{association list}.

\label Description::

%% 15.6.0 5
Creates a \term{fresh} \term{cons},
the \term{cdr} of which is \param{alist} and
the \term{car} of which is another \term{fresh} \term{cons},
 the \term{car} of which is \param{key} and
 the \term{cdr} of which is \param{datum}.

\label Examples::

\code
 (setq alist '()) \EV NIL
 (acons 1 "one" alist) \EV ((1 . "one"))
 alist \EV NIL
 (setq alist (acons 1 "one" (acons 2 "two" alist))) \EV ((1 . "one") (2 . "two"))
 (assoc 1 alist) \EV (1 . "one")
 (setq alist (acons 1 "uno" alist)) \EV ((1 . "uno") (1 . "one") (2 . "two"))
 (assoc 1 alist) \EV (1 . "uno")
\endcode


\label Side Effects:\None.

\label Affected By:\None.

\label Exceptional Situations:\None.
 
\label See Also::

\funref{assoc}, \funref{pairlis}

\label Notes::

\code
(acons \param{key} \param{datum} \param{alist}) \EQ (cons (cons \param{key} \param{datum}) \param{alist})
\endcode

\endcom

%%% ========== ASSOC
%%% ========== ASSOC-IF
%%% ========== ASSOC-IF-NOT
\begincom{assoc, assoc-if, assoc-if-not}\ftype{Function}

\label Syntax::

\DefunWithValues assoc        {item      alist {\key} key test test-not} {entry}

\issue{ASSOC-RASSOC-IF-KEY:YES}

\DefunWithValues assoc-if     {predicate alist {\key} key} {entry}
\DefunWithValues assoc-if-not {predicate alist {\key} key} {entry}

\endissue{ASSOC-RASSOC-IF-KEY:YES}

\label Arguments and Values::

\param{item}---an \term{object}.

\param{alist}---an \term{association list}.

\param{predicate}---a \term{designator} for 
		    a \term{function} of one \term{argument}
		    that returns a \term{generalized boolean}.

\param{test}---a \term{designator} for a \term{function} of two \term{arguments}
  that returns a \term{generalized boolean}.

\param{test-not}---a \term{designator} for 
  a \term{function} of two \term{arguments}
  that returns a \term{generalized boolean}.

\param{key}---a \term{designator} for a \term{function} of one argument,
  or \nil.

\param{entry}---a \term{cons} that is an \term{element} of \param{alist},
	        or \nil.

\label Description::

\issue{ASSOC-RASSOC-IF-KEY}
%% Replaced per X3J13. -kmp 05-Oct-93
% %% 15.6.0 8
% \funref{assoc} returns the first \term{cons} in \param{alist} whose 
% \term{car} \term{satisfies the test}, or \nil\ if no such \term{cons} is found.
% 
% The arguments to \param{test} and \param{test-not} and are the key of the
% \param{item} and the key of the \term{car} of an element of \param{alist}, 
% in that order.
% 
% \funref{assoc-if} and \funref{assoc-if-not} return the first \term{cons} 
% in \param{alist} whose \term{car} \term{satisfies the predicate}, or \nil\ if 
% no such \term{cons} is found.
% 
% The argument to \param{predicate} is the key of an element of \param{alist}.
\funref{assoc}, \funref{assoc-if}, and \funref{assoc-if-not}
return the first \term{cons} in \param{alist} whose \term{car} \term{satisfies the test},
or \nil\ if no such \term{cons} is found.
\endissue{ASSOC-RASSOC-IF-KEY}
                    
For \funref{assoc}, \funref{assoc-if}, and \funref{assoc-if-not}, if \nil\ appears
in \param{alist} in place of a pair, it is ignored.

\label Examples::
%% 15.6.0 9

\issue{ASSOC-RASSOC-IF-KEY}
\code
 (setq values '((x . 100) (y . 200) (z . 50))) \EV ((X . 100) (Y . 200) (Z . 50))
 (assoc 'y values) \EV (Y . 200)
 (rplacd (assoc 'y values) 201) \EV (Y . 201)
 (assoc 'y values) \EV (Y . 201)
 (setq alist '((1 . "one")(2 . "two")(3 . "three"))) 
\EV ((1 . "one") (2 . "two") (3 . "three"))
 (assoc 2 alist) \EV (2 . "two")
 (assoc-if #'evenp alist) \EV (2 . "two")
 (assoc-if-not #'(lambda(x) (< x 3)) alist) \EV (3 . "three")
 (setq alist '(("one" . 1)("two" . 2))) \EV (("one" . 1) ("two" . 2))
 (assoc "one" alist) \EV NIL
 (assoc "one" alist :test #'equalp) \EV ("one" . 1)
 (assoc "two" alist :key #'(lambda(x) (char x 2))) \EV NIL 
 (assoc #\\o alist :key #'(lambda(x) (char x 2))) \EV ("two" . 2)
 (assoc 'r '((a . b) (c . d) (r . x) (s . y) (r . z))) \EV  (R . X)
 (assoc 'goo '((foo . bar) (zoo . goo))) \EV NIL
 (assoc '2 '((1 a b c) (2 b c d) (-7 x y z))) \EV (2 B C D)
 (setq alist '(("one" . 1) ("2" . 2) ("three" . 3)))
\EV (("one" . 1) ("2" . 2) ("three" . 3))
 (assoc-if-not #'alpha-char-p alist
               :key #'(lambda (x) (char x 0))) \EV ("2" . 2)
\endcode
\endissue{ASSOC-RASSOC-IF-KEY}

\label Affected By:\None.

\label Exceptional Situations::

\Lazychecktype{alist}{an \term{association list}}

\label See Also::

\funref{rassoc},
\funref{find},
\funref{member},
\funref{position},
\issue{MAPPING-DESTRUCTIVE-INTERACTION:EXPLICITLY-VAGUE}
{\secref\TraversalRules}
\endissue{MAPPING-DESTRUCTIVE-INTERACTION:EXPLICITLY-VAGUE}

\label Notes::

\issue{TEST-NOT-IF-NOT:FLUSH-ALL}
The \kwd{test-not} parameter is deprecated.
\endissue{TEST-NOT-IF-NOT:FLUSH-ALL}

\issue{TEST-NOT-IF-NOT:FLUSH-ALL}
\Thefunction{assoc-if-not} is deprecated.
\endissue{TEST-NOT-IF-NOT:FLUSH-ALL}

It is possible to \funref{rplacd} the result of \funref{assoc}, provided
that it is not \nil,
in order to ``update'' \param{alist}. 

%% 15.6.0 10
The two expressions

\code
 (assoc item list :test fn)
\endcode
and

\code
 (find item list :test fn :key #'car)
\endcode
are equivalent in meaning with one exception:
if \nil\ appears in \param{alist} in place of a pair,
and \param{item} is \nil,
\funref{find} will compute the \term{car} of the \nil\ in \param{alist},
find that it is equal to \param{item}, and return \nil,
whereas \funref{assoc} will ignore the \nil\ in \param{alist} and continue
to search for an actual \term{cons} whose \term{car} is \nil.

\endcom


%%% ========== COPY-ALIST
\begincom{copy-alist}\ftype{Function}

\label Syntax::

\DefunWithValues copy-alist {alist} {new-alist}

\label Arguments and Values::

\param{alist}---an \term{association list}.

\param{new-alist}---an \term{association list}.

\label Description::

%% 15.2.0 24
\funref{copy-alist} returns a \term{copy} of \param{alist}.

The \term{list structure} of \param{alist} is copied,
and the \term{elements} of \param{alist} which are \term{conses} are
also copied (as \term{conses} only). 
Any other \term{objects} which are referred to, 
whether directly or indirectly,
by the \param{alist} continue to be shared.

\label Examples::

\code
(defparameter *alist* (acons 1 "one" (acons 2 "two" '())))
*alist* \EV ((1 . "one") (2 . "two"))
(defparameter *list-copy* (copy-list *alist*))
*list-copy* \EV ((1 . "one") (2 . "two"))
(defparameter *alist-copy* (copy-alist *alist*))
*alist-copy* \EV ((1 . "one") (2 . "two"))
(setf (cdr (assoc 2 *alist-copy*)) "deux") \EV "deux"
*alist-copy* \EV ((1 . "one") (2 . "deux"))
*alist* \EV ((1 . "one") (2 . "two"))
(setf (cdr (assoc 1 *list-copy*)) "uno") \EV "uno"
*list-copy* \EV ((1 . "uno") (2 . "two"))
*alist* \EV ((1 . "uno") (2 . "two"))
\endcode

\label Side Effects:\None.

\label Affected By:\None.

\label Exceptional Situations:\None.

\label See Also::

\funref{copy-list}

\label Notes:\None.

\endcom

%%% ========== PAIRLIS
\begincom{pairlis}\ftype{Function}

\label Syntax::

\DefunWithValues pairlis {keys data {\opt} alist} {new-alist}

\label Arguments and Values::

\param{keys}---a \term{proper list}.

\param{data}---a \term{proper list}.

\param{alist}---an \term{association list}.
  \Default{the \term{empty list}}

\param{new-alist}---an \term{association list}.

\label Description::

%% 15.6.0 6
Returns an \term{association list} that associates
elements of \param{keys} to corresponding elements of \param{data}.
The consequences are undefined if \param{keys} and \param{data} are 
not of the same \term{length}.  

If \param{alist} is supplied, \funref{pairlis} returns
a modified \param{alist} with the
new pairs prepended to it.
%% 15.6.0 7
The new pairs may appear in the resulting \term{association list} in 
either forward or backward order.
The result of 

\code
 (pairlis '(one two) '(1 2) '((three . 3) (four . 19)))
\endcode
might be

\code
 ((one . 1) (two . 2) (three . 3) (four . 19))
\endcode
or

\code
 ((two . 2) (one . 1) (three . 3) (four . 19))
\endcode


\label Examples::
\code
 (setq keys '(1 2 3)
        data '("one" "two" "three")
        alist '((4 . "four"))) \EV ((4 . "four"))
 (pairlis keys data) \EV ((3 . "three") (2 . "two") (1 . "one"))
 (pairlis keys data alist)
\EV ((3 . "three") (2 . "two") (1 . "one") (4 . "four"))
 alist \EV ((4 . "four"))
\endcode


\label Side Effects:\None.

\label Affected By:\None.

\label Exceptional Situations::

\Lazychecktypes{\param{keys} and \param{data}}{\term{proper lists}}

\label See Also::

\funref{acons}

\label Notes:\None.

\endcom

%%% ========== RASSOC
%%% ========== RASSOC-IF
%%% ========== RASSOC-IF-NOT
\begincom{rassoc, rassoc-if, rassoc-if-not}\ftype{Function}

\label Syntax::

\DefunWithValues rassoc        {item      alist  {\key} key test test-not} {entry}

\issue{ASSOC-RASSOC-IF-KEY:YES}

\DefunWithValues rassoc-if     {predicate alist {\key} key} {entry}
\DefunWithValues rassoc-if-not {predicate alist {\key} key} {entry}

\endissue{ASSOC-RASSOC-IF-KEY:YES}

\label Arguments and Values:: 

\param{item}---an \term{object}.

\param{alist}---an \term{association list}.

\param{predicate}---a \term{designator} for
		    a \term{function} of one \term{argument}
		    that returns a \term{generalized boolean}.

\param{test}---a \term{designator} for a \term{function} of two \term{arguments}
  that returns a \term{generalized boolean}.

\param{test-not}---a \term{designator} for 
  a \term{function} of two \term{arguments}
  that returns a \term{generalized boolean}.

\param{key}---a \term{designator} for a \term{function} of one argument,
  or \nil.

\param{entry}---a \term{cons} that is an \term{element} of the \param{alist},
		or \nil.

\label Description::

%% 15.6.0 13
\funref{rassoc}, \funref{rassoc-if}, and \funref{rassoc-if-not}
return the first \term{cons} whose \term{cdr} 
\term{satisfies the test}.
If no such \term{cons} is found, \nil\
is returned.

\issue{ASSOC-RASSOC-IF-KEY}
%% Removed per X3J13. -kmp 05-Oct-93
% The argument to \param{predicate} is an element of \param{alist}
% as returned by the \kwd{key} function (if supplied).
% The arguments to \kwd{test} and \kwd{test-not}
% are \param{item} and an element of \param{alist}
% as returned by the \kwd{key} function (if supplied), in that
% order.
% 
% If \kwd{key} is supplied, it is applied to the \term{cdr}
% of the \param{alist} entries and the result is passed to the
% \param{predicate}, \kwd{test}, or \kwd{test-not} function.
% The \term{cdr} of the \param{alist} entry contains the
% key of the association, and if \kwd{key} is
% not supplied or \nil, the \term{cdr} is the key of the association.
\endissue{ASSOC-RASSOC-IF-KEY}

If \nil\ appears in \param{alist} in place of a pair, it is  ignored.

\label Examples::

\code
 (setq alist '((1 . "one") (2 . "two") (3 . 3))) 
\EV ((1 . "one") (2 . "two") (3 . 3))
 (rassoc 3 alist) \EV (3 . 3)
 (rassoc "two" alist) \EV NIL
 (rassoc "two" alist :test 'equal) \EV (2 . "two")
 (rassoc 1 alist :key #'(lambda (x) (if (numberp x) (/ x 3)))) \EV (3 . 3)
 (rassoc 'a '((a . b) (b . c) (c . a) (z . a))) \EV (C . A)
 (rassoc-if #'stringp alist) \EV (1 . "one")
 (rassoc-if-not #'vectorp alist) \EV (3 . 3)
\endcode

\label Side Effects:\None.

\label Affected By:\None.

\label Exceptional Situations:\None.

\label See Also::

\funref{assoc},
\issue{MAPPING-DESTRUCTIVE-INTERACTION:EXPLICITLY-VAGUE}
{\secref\TraversalRules}
\endissue{MAPPING-DESTRUCTIVE-INTERACTION:EXPLICITLY-VAGUE}

\label Notes::

\issue{TEST-NOT-IF-NOT:FLUSH-ALL}
The \kwd{test-not} parameter is deprecated.
\endissue{TEST-NOT-IF-NOT:FLUSH-ALL}

\issue{TEST-NOT-IF-NOT:FLUSH-ALL}
\Thefunction{rassoc-if-not} is deprecated.
\endissue{TEST-NOT-IF-NOT:FLUSH-ALL}

It is possible to \funref{rplaca} the result of \funref{rassoc}, 
provided that it is not \nil, in order to ``update'' \param{alist}. 

%% 15.6.0 13

The expressions

\code
 (rassoc item list :test fn)
\endcode
and

\code
 (find item list :test fn :key #'cdr)
\endcode
are equivalent in meaning, except when the \f{item} is \nil\
and \nil\ appears in place of a pair in the \param{alist}.
\Seefun{assoc}. 

\endcom



%-------------------- Plists --------------------

%%% ========== GET-PROPERTIES
\begincom{get-properties}\ftype{Function}

\label Syntax::

\DefunWithValues get-properties {plist indicator-list} {indicator, value, tail}

\label Arguments and Values:: 

\issue{PLIST-DUPLICATES:ALLOW}
\param{plist}---a \term{property list}.
\endissue{PLIST-DUPLICATES:ALLOW}

\param{indicator-list}---a \term{proper list} (of \term{indicators}).

\param{indicator}---an \term{object} that is an \term{element} of \param{indicator-list}.

\param{value}---an \term{object}.

\param{tail}---a \term{list}.

\label Description::

%% 10.1.0 19
\funref{get-properties} is used to look up any of several
\term{property list} entries all at once.

%% 10.1.0 23        
It searches the \param{plist} for the first entry whose \term{indicator} 
is \term{identical} to one of the \term{objects} in \param{indicator-list}.
If such an entry is found, the \param{indicator} and \param{value} returned
are the \term{property indicator} and its associated \term{property value},
and the \param{tail} returned is the \term{tail} of the \param{plist}
that begins with the found entry (\ie whose \term{car} is the \param{indicator}).
If no such entry is found, the \param{indicator}, \param{value}, and \param{tail} 
are all \nil.
                                              
\label Examples::

\code
 (setq x '()) \EV NIL
 (setq *indicator-list* '(prop1 prop2)) \EV (PROP1 PROP2)
 (getf x 'prop1) \EV NIL
 (setf (getf x 'prop1) 'val1) \EV VAL1
 (eq (getf x 'prop1) 'val1) \EV \term{true}
 (get-properties x *indicator-list*) \EV PROP1, VAL1, (PROP1 VAL1)
 x \EV (PROP1 VAL1)
\endcode

\label Side Effects:\None.

\label Affected By:\None.

\label Exceptional Situations:\None.

\label See Also::

\funref{get}, \funref{getf}

\label Notes:\None.

\endcom

%%% ========== GETF
\begincom{getf}\ftype{Accessor}

\label Syntax::

\DefunWithValues getf {plist indicator {\opt} default} {value}
\Defsetf         getf {place indicator {\opt} default} {new-value}

\label Arguments and Values::

\param{plist}---a \term{property list}.

\param{place}---a \term{place}, the \term{value} of which is a \term{property list}.

\param{indicator}---an \term{object}.

\param{default}---an \term{object}.
 \Default{\nil}

%% 10.1.0 24
\param{value}---an \term{object}.

\param{new-value}---an \term{object}.

\label Description::

%% 10.1.0 19
% \funref{getf} is used to \term{access} entries in a \term{property list}.
% \funref{getf} searches \param{plist} for an indicator 
% \term{identical} to \param{indicator} and returns the associated value.
\funref{getf} finds a \term{property} on the \param{plist}
whose \term{property indicator} is \term{identical} to \param{indicator},
and returns its corresponding \term{property value}.
\issue{PLIST-DUPLICATES:ALLOW}
If there are multiple \term{properties}\meaning{1} with that \term{property indicator},
\funref{getf} uses the first such \term{property}.
\endissue{PLIST-DUPLICATES:ALLOW}
% If \param{indicator} is not found, then \param{default} is returned.
If there is no \term{property} with that \term{property indicator},
\param{default} is returned.

% %% 10.1.0 20
% For \macref{setf} of \funref{getf}, the effect is 
%    to add a new property-value pair,
% or to update an existing pair,
% in the \term{property list} that is the \term{value} of \param{place}.
\macref{setf} of \funref{getf} may be used to associate a new \term{object}
with an existing indicator in the \term{property list} held by \param{place},
or to create a new assocation if none exists.
\issue{PLIST-DUPLICATES:ALLOW}
If there are multiple \term{properties}\meaning{1} with that \term{property indicator},
\macref{setf} of \funref{getf} associates the \param{new-value} 
with the first such \term{property}.
\endissue{PLIST-DUPLICATES:ALLOW}
\issue{SETF-GET-DEFAULT:EVALUATED-BUT-IGNORED}
When a \funref{getf} \term{form} is used as a \macref{setf} \param{place},
any \param{default} which is supplied is evaluated according to normal
left-to-right evaluation rules, but its \term{value} is ignored.
\endissue{SETF-GET-DEFAULT:EVALUATED-BUT-IGNORED}

%% 10.1.0 20
\issue{REMF-DESTRUCTION-UNSPECIFIED:X3J13-MAR-89}
\macref{setf} of \funref{getf} is permitted to either
    \term{write} the \term{value} of \param{place} itself,
 or modify of any part, \term{car} or \term{cdr}, 
     of the \term{list structure} held by \param{place}.
\endissue{REMF-DESTRUCTION-UNSPECIFIED:X3J13-MAR-89}%

\label Examples::

\issue{REMF-DESTRUCTION-UNSPECIFIED:X3J13-MAR-89}
\code
 (setq x '()) \EV NIL
 (getf x 'prop1) \EV NIL
 (getf x 'prop1 7) \EV 7
 (getf x 'prop1) \EV NIL
 (setf (getf x 'prop1) 'val1) \EV VAL1
 (eq (getf x 'prop1) 'val1) \EV \term{true}
 (getf x 'prop1) \EV VAL1
 (getf x 'prop1 7) \EV VAL1
 x \EV (PROP1 VAL1)

;; Examples of implementation variation permitted.
 (setq foo (list 'a 'b 'c 'd 'e 'f)) \EV (A B C D E F)
 (setq bar (cddr foo)) \EV (C D E F)
 (remf foo 'c) \EV \term{true}
 foo \EV (A B E F)
 bar
\EV (C D E F)
\OV (C)
\OV (NIL)
\OV (C NIL)
\OV (C D)
\endcode
\endissue{REMF-DESTRUCTION-UNSPECIFIED:X3J13-MAR-89}
 
\label Side Effects:\None.

\label Affected By:\None.

\label Exceptional Situations:\None.

\label See Also::

\funref{get},
\funref{get-properties},
\macref{setf},
{\secref\FnFormsAsGenRefs}

\label Notes::

There is no way (using \funref{getf}) to distinguish an absent property
from one whose value is \param{default}; but see \funref{get-properties}.

Note that while supplying a \term{default} argument to \macref{getf}
in a \macref{setf} situation is sometimes not very interesting,
it is still important because some macros, such as \macref{push} and
\macref{incf}, require a \param{place} argument which data is both \term{read}
from and \term{written} to.  In such a context, if a \term{default} 
argument is to be supplied for the \term{read} situation, it must be
syntactically valid for the \term{write} situation as well. For example,

\code
 (let ((plist '()))
   (incf (getf plist 'count 0))
   plist) \EV (COUNT 1)
\endcode

\endcom

%%% ========== REMF
\begincom{remf}\ftype{Macro}

\label Syntax::

\DefmacWithValues remf {place indicator} {generalized-boolean}

\label Arguments and Values::

\param{place}---a \term{place}.

\param{indicator}---an \term{object}.

\param{generalized-boolean}---a \term{generalized boolean}.

\label Description::

%% 10.1.0 22
\macref{remf} removes from the \term{property list} stored in \param{place}
a \term{property}\meaning{1} with a \term{property indicator}
% EQ => identical -kmp 14-Jul-93
\term{identical} to \param{indicator}.
\issue{PLIST-DUPLICATES:ALLOW}
If there are multiple \term{properties}\meaning{1} with the \term{identical} key,
\macref{remf} only removes the first such \term{property}.
\endissue{PLIST-DUPLICATES:ALLOW}
\macref{remf} returns \term{false} if no such \term{property} was found,
or \term{true} if a property was found.

The \term{property indicator} 
and the corresponding \term{property value} 
are removed in an undefined order
by destructively splicing the property list.  
\issue{REMF-DESTRUCTION-UNSPECIFIED:X3J13-MAR-89}
\macref{remf} is permitted to either \macref{setf} \param{place} or to 
\macref{setf} any part, \funref{car} or \funref{cdr}, 
of the \term{list structure} held by that \param{place}.
\endissue{REMF-DESTRUCTION-UNSPECIFIED:X3J13-MAR-89}

\issue{PUSH-EVALUATION-ORDER:FIRST-ITEM}
For information about the \term{evaluation} of \term{subforms} of \param{place},
\seesection\GenRefSubFormEval.
\endissue{PUSH-EVALUATION-ORDER:FIRST-ITEM}

\label Examples::

\code
 (setq x (cons () ())) \EV (NIL)
 (setf (getf (car x) 'prop1) 'val1) \EV VAL1
 (remf (car x) 'prop1) \EV \term{true}
 (remf (car x) 'prop1) \EV \term{false}
\endcode

\label Side Effects::

The property list stored in \param{place} is modified.

\label Affected By:\None.

\label Exceptional Situations:\None.

\label See Also::

\funref{remprop}, \funref{getf}

\label Notes:\None.

\endcom

%-------------------- Sets --------------------

%%% ========== INTERSECTION
%%% ========== NINTERSECTION
\begincom{intersection, nintersection}\ftype{Function}

\label Syntax::

\DefunWithValues intersection  {list-1 list-2 {\key} key test test-not} {result-list}
\DefunWithValues nintersection {list-1 list-2 {\key} key test test-not} {result-list}

\label Arguments and Values:: 

\param{list-1}---a \term{proper list}.

\param{list-2}---a \term{proper list}.

%% 15.5.0 15
\param{test}---a \term{designator} for a \term{function} of two \term{arguments}
  that returns a \term{generalized boolean}.

\param{test-not}---a \term{designator} for 
  a \term{function} of two \term{arguments}
  that returns a \term{generalized boolean}.

\param{key}---a \term{designator} for a \term{function} of one argument,
  or \nil.

\param{result-list}---a \term{list}.
     
\label Description::

%% 15.5.0 14
\funref{intersection} and \funref{nintersection} return a \term{list}
that contains every element that occurs in both \param{list-1} and \param{list-2}.

%% 15.5.0 16
\funref{nintersection} is the destructive version of \funref{intersection}.
It performs the same operation,
but may destroy \param{list-1} using its cells to construct the result.
\issue{NINTERSECTION-DESTRUCTION:REVERT}
\issue{REMF-DESTRUCTION-UNSPECIFIED:X3J13-MAR-89}
%% This has gone back and forth, but my opinion is that the current state of this
%% is that the second arg is again protected. -kmp 7-Feb-92
%%
%% This opinion finally validated by clarifying vote in letter ballot (X3J13/93-302).
%% Option REVERT affirms the status quo from both CLtL1 and draft 12.24,
%% overriding the possible effect of REMF-DESTRUCTION-UNSPECIFIED:X3J13-MAR-89.
%
% %%Barmar noted that this next sentence was in conflict with the changes
% %%made by REMF-DESTRUCTION-UNSPECIFIED, so I've removed it for now.  Mail sent
% %%to find out if X3J13 really intended for NINTERSECTION's contract to be changed,
% %%or if making this untrue this was a `typo':
% %\param{list-2} is not destroyed.
% 
% \funref{nintersection} is permitted to modify any part of the \term{list structure} 
% of \param{list-1} or \param{list-2}.
%
\param{list-2} is not destroyed.
\endissue{REMF-DESTRUCTION-UNSPECIFIED:X3J13-MAR-89}
\endissue{NINTERSECTION-DESTRUCTION:REVERT}

The intersection operation is described as follows.
For all possible ordered pairs consisting of 
    one \term{element} from \param{list-1} 
and one \term{element} from \param{list-2},
\kwd{test} or \kwd{test-not} are used 
to determine whether they \term{satisfy the test}.  
The first argument to the \kwd{test} or \kwd{test-not}
function is an element of \param{list-1}; the second argument is an
element of \param{list-2}.
If \kwd{test} or \kwd{test-not} is not supplied, \funref{eql}
is used.
It is an error if \kwd{test} and \kwd{test-not} are supplied in
the same function call.

If \kwd{key} is supplied (and not \nil), it is used to
extract the part to be tested from the \param{list} element. 
The argument to the \kwd{key} function
is an element of either \param{list-1} or \param{list-2};
the \kwd{key} function typically returns part of the supplied element.
If \kwd{key} is not supplied or \nil, the \param{list-1} and
\param{list-2} elements are used.

For every pair that \term{satifies the test},
exactly one of the two elements of the pair will be put in the result.
No element from either \term{list} appears in the result that does not 
\term{satisfy the test} for
an element from the other \term{list}.
If one of the \term{lists} contains duplicate
elements, there may be duplication in the result.

There is no guarantee that the order of elements in the result will
reflect the ordering of the arguments in any particular way.
The result \term{list} may share cells with, 
or be \funref{eq} to, either \param{list-1} or \param{list-2}
if appropriate.

\label Examples::

\code
 (setq list1 (list 1 1 2 3 4 a b c "A" "B" "C" "d")
       list2 (list 1 4 5 b c d "a" "B" "c" "D")) 
  \EV (1 4 5 B C D "a" "B" "c" "D")
 (intersection list1 list2) \EV (C B 4 1 1)
 (intersection list1 list2 :test 'equal) \EV ("B" C B 4 1 1)
 (intersection list1 list2 :test #'equalp) \EV ("d" "C" "B" "A" C B 4 1 1) 
 (nintersection list1 list2) \EV (1 1 4 B C)
 list1 \EV \term{implementation-dependent} ;\eg (1 1 4 B C)
 list2 \EV \term{implementation-dependent} ;\eg (1 4 5 B C D "a" "B" "c" "D")
 (setq list1 (copy-list '((1 . 2) (2 . 3) (3 . 4) (4 . 5))))
\EV ((1 . 2) (2 . 3) (3 . 4) (4 . 5)) 
 (setq list2 (copy-list '((1 . 3) (2 . 4) (3 . 6) (4 . 8))))
\EV ((1 . 3) (2 . 4) (3 . 6) (4 . 8)) 
 (nintersection list1 list2 :key #'cdr) \EV ((2 . 3) (3 . 4)) 
 list1 \EV \term{implementation-dependent} ;\eg ((1 . 2) (2 . 3) (3 . 4)) 
 list2 \EV \term{implementation-dependent} ;\eg ((1 . 3) (2 . 4) (3 . 6) (4 . 8)) 
\endcode

\label Side Effects::

\issue{REMF-DESTRUCTION-UNSPECIFIED:X3J13-MAR-89}
\funref{nintersection} can modify \param{list-1}, 
\issue{NINTERSECTION-DESTRUCTION}
%or \param{list-2}.
but not \param{list-2}.
\endissue{NINTERSECTION-DESTRUCTION}
\endissue{REMF-DESTRUCTION-UNSPECIFIED:X3J13-MAR-89}

\label Affected By:\None.

\label Exceptional Situations::

\Lazychecktypes{\param{list-1} and \param{list-2}}{\term{proper lists}}

\label See Also::

\funref{union},
\issue{CONSTANT-MODIFICATION:DISALLOW}
{\secref\ConstantModification},
\endissue{CONSTANT-MODIFICATION:DISALLOW}
\issue{MAPPING-DESTRUCTIVE-INTERACTION:EXPLICITLY-VAGUE}
{\secref\TraversalRules}
\endissue{MAPPING-DESTRUCTIVE-INTERACTION:EXPLICITLY-VAGUE}

\label Notes::

\issue{TEST-NOT-IF-NOT:FLUSH-ALL}
The \kwd{test-not} parameter is deprecated.
\endissue{TEST-NOT-IF-NOT:FLUSH-ALL}

\issue{REMF-DESTRUCTION-UNSPECIFIED:X3J13-MAR-89}
Since the \funref{nintersection} side effect is not required,
it should not be used in for-effect-only
  positions in portable code.
\endissue{REMF-DESTRUCTION-UNSPECIFIED:X3J13-MAR-89}
\endcom

%%% ========== ADJOIN
\begincom{adjoin}\ftype{Function}

\label Syntax::

\DefunWithValues adjoin {item list {\key} key test test-not} {new-list}

\label Arguments and Values:: 

\param{item}---an \term{object}.

\param{list}---a \term{proper list}.

\param{test}---a \term{designator} for a \term{function} of two \term{arguments}
  that returns a \term{generalized boolean}.

\param{test-not}---a \term{designator} for 
  a \term{function} of two \term{arguments}
  that returns a \term{generalized boolean}.

\param{key}---a \term{designator} for a \term{function} of one argument,
  or \nil.

%% 15.5.0 8   
\param{new-list}---a \term{list}.

\label Description::

Tests whether \param{item} is the same as an existing element of \param{list}.
If the \param{item} is not an existing element,
\funref{adjoin} adds it to \param{list} (as if by \funref{cons})
and returns the resulting \term{list}; 
otherwise, nothing is added and the original \param{list} is returned.

\SatisfyTest{item}{an \term{element} of \param{list}}

\label Examples::

\code
 (setq slist '()) \EV NIL 
 (adjoin 'a slist) \EV (A) 
 slist \EV NIL 
 (setq slist (adjoin '(test-item 1) slist)) \EV ((TEST-ITEM 1)) 
 (adjoin '(test-item 1) slist) \EV ((TEST-ITEM 1) (TEST-ITEM 1)) 
 (adjoin '(test-item 1) slist :test 'equal) \EV ((TEST-ITEM 1)) 
 (adjoin '(new-test-item 1) slist :key #'cadr) \EV ((TEST-ITEM 1)) 
 (adjoin '(new-test-item 1) slist) \EV ((NEW-TEST-ITEM 1) (TEST-ITEM 1)) 
\endcode

\label Affected By:\None.

\label Exceptional Situations::

\Lazychecktype{list}{a \term{proper list}}

\label See Also::

\macref{pushnew},
\issue{MAPPING-DESTRUCTIVE-INTERACTION:EXPLICITLY-VAGUE}
{\secref\TraversalRules}
\endissue{MAPPING-DESTRUCTIVE-INTERACTION:EXPLICITLY-VAGUE}

\label Notes::

\issue{TEST-NOT-IF-NOT:FLUSH-ALL}
The \kwd{test-not} parameter is deprecated.
\endissue{TEST-NOT-IF-NOT:FLUSH-ALL}

\code
 (adjoin item list :key fn)
   \EQ (if (member (fn item) list :key fn) list (cons item list))
\endcode
      
\endcom

%%% ========== PUSHNEW
\begincom{pushnew}\ftype{Macro}

\label Syntax::

\DefmacWithValuesNewline pushnew {item place {\key} key test test-not} {new-place-value}

\label Arguments and Values:: 

\param{item}---an \term{object}.  

%!!! Need to separate discussion of place from its value. -kmp 1-Sep-91
\param{place}---a \term{place}, the \term{value} of which is a \term{proper list}. 

\param{test}---a \term{designator} for a \term{function} of two \term{arguments}
  that returns a \term{generalized boolean}.

\param{test-not}---a \term{designator} for 
  a \term{function} of two \term{arguments}
  that returns a \term{generalized boolean}.

\param{key}---a \term{designator} for a \term{function} of one argument,
  or \nil.

\param{new-place-value}---a \term{list} (the new \term{value} of \param{place}).

\label Description::

\macref{pushnew} tests whether  \param{item} is the same as any existing
element of the \term{list} stored in \param{place}.  If \param{item} is not,
it is prepended to the \term{list}, and the new \term{list} is stored in
\param{place}.

%% 15.2.0 34
\macref{pushnew} returns the new \term{list} that is stored in \param{place}.

%% 15.2.0 32
Whether or not \param{item} is already a member of the \term{list} that is
in \param{place} is determined by comparisons using \kwd{test} or \kwd{test-not}.
The first argument to the \kwd{test} or \kwd{test-not}
function is \param{item}; the second argument is
an element of the \term{list} in \param{place} as returned by
the \kwd{key} function (if supplied).

If \kwd{key} is supplied, it is used to extract the part to be tested from
both \param{item} and the \term{list} element,
as for \funref{adjoin}.

The argument to the \kwd{key} function 
is an element of the \term{list} stored in 
\param{place}. The \kwd{key} function typically returns part
part of the element of the \term{list}.
If \kwd{key} is not supplied or \nil, the \term{list} 
element is used.

\issue{PUSH-EVALUATION-ORDER:FIRST-ITEM}
For information about the \term{evaluation} of \term{subforms} of \param{place},
\seesection\GenRefSubFormEval.
\endissue{PUSH-EVALUATION-ORDER:FIRST-ITEM}

\issue{PUSHNEW-STORE-REQUIRED:UNSPECIFIED}
It is \term{implementation-dependent} whether or not \macref{pushnew} 
actually executes the storing form for its \param{place} in the
situation where the \param{item} is already a member of the \term{list}
held by \param{place}.
\endissue{PUSHNEW-STORE-REQUIRED:UNSPECIFIED}

\label Examples::
\code
 (setq x '(a (b c) d)) \EV (A (B C) D)
 (pushnew 5 (cadr x)) \EV (5 B C)   
 x \EV (A (5 B C) D)
 (pushnew 'b (cadr x)) \EV (5 B C)  
 x \EV (A (5 B C) D)
 (setq lst '((1) (1 2) (1 2 3))) \EV ((1) (1 2) (1 2 3))
 (pushnew '(2) lst) \EV ((2) (1) (1 2) (1 2 3))
 (pushnew '(1) lst) \EV ((1) (2) (1) (1 2) (1 2 3))
 (pushnew '(1) lst :test 'equal) \EV ((1) (2) (1) (1 2) (1 2 3))
 (pushnew '(1) lst :key #'car) \EV ((1) (2) (1) (1 2) (1 2 3)) 
\endcode

\label Side Effects::

The contents of \param{place} may be modified.

\label Affected By:\None.

\label Exceptional Situations:\None.

\label See Also::

\macref{push},
\funref{adjoin},
{\secref\GeneralizedReference}

\label Notes::

The effect of
\code
 (pushnew item place :test p)
\endcode
is roughly equivalent to
\code
 (setf place (adjoin item place :test p))
\endcode
\issue{PUSH-EVALUATION-ORDER:FIRST-ITEM}
except that the \term{subforms} of \f{place} are evaluated only once, 
and \f{item} is evaluated before \f{place}.
\endissue{PUSH-EVALUATION-ORDER:FIRST-ITEM}

\endcom

%%% ========== NSET-DIFFERENCE
%%% ========== SET-DIFFERENCE
\begincom{set-difference, nset-difference}\ftype{Function}

\label Syntax::

\DefunWithValues set-difference  {list-1 list-2 {\key} key test test-not} {result-list}
\DefunWithValues nset-difference {list-1 list-2 {\key} key test test-not} {result-list}

\label Arguments and Values:: 

\param{list-1}---a \term{proper list}.

\param{list-2}---a \term{proper list}.

\param{test}---a \term{designator} for a \term{function} of two \term{arguments}
  that returns a \term{generalized boolean}.

\param{test-not}---a \term{designator} for 
  a \term{function} of two \term{arguments}
  that returns a \term{generalized boolean}.

\param{key}---a \term{designator} for a \term{function} of one argument,
  or \nil.

\param{result-list}---a \term{list}.
                                   
\label Description::
%% 15.5.0 17
\funref{set-difference} returns a \term{list} 
of elements of \param{list-1}
that do not appear in \param{list-2}.  

%% 15.5.0 20
\funref{nset-difference} is the destructive 
version of \funref{set-difference}.
It may destroy \param{list-1}.

%% 15.5.0 19
For all possible ordered pairs consisting of
one element from \param{list-1} and one element from \param{list-2}, the 
\kwd{test} or \kwd{test-not} function is
used to determine whether they \term{satisfy the test}.  
The first argument to the \kwd{test} or \kwd{test-not} function 
is the part of an element of \param{list-1} that is returned by 
the \kwd{key} function (if supplied); the second argument is the part of 
an element of \param{list-2} that is 
returned by the \kwd{key} function (if supplied).

If \kwd{key} is supplied, its argument is a \param{list-1} or
\param{list-2} element. The \kwd{key} function 
typically returns part of 
the supplied element.
If \kwd{key} is not supplied, the \param{list-1} or \param{list-2}
element is used.

An element of \param{list-1}
appears in the result if and only if it does not match any element
of \param{list-2}.  


%% 15.5.0 18
There is no guarantee that the order of elements in the result will
reflect the ordering of the arguments in any particular way.
The result \term{list} 
may share cells with, or be \funref{eq} to, either of \param{list-1}
or \param{list-2},
if appropriate.

\label Examples::

\code
 (setq lst1 (list "A" "b" "C" "d")
       lst2 (list "a" "B" "C" "d")) \EV ("a" "B" "C" "d")
 (set-difference lst1 lst2) \EV ("d" "C" "b" "A")
 (set-difference lst1 lst2 :test 'equal) \EV ("b" "A")
 (set-difference lst1 lst2 :test #'equalp) \EV NIL 
 (nset-difference lst1 lst2 :test #'string=) \EV ("A" "b")
 (setq lst1 '(("a" . "b") ("c" . "d") ("e" . "f")))
\EV (("a" . "b") ("c" . "d") ("e" . "f")) 
 (setq lst2 '(("c" . "a") ("e" . "b") ("d" . "a")))
\EV (("c" . "a") ("e" . "b") ("d" . "a")) 
 (nset-difference lst1 lst2 :test #'string= :key #'cdr)
\EV (("c" . "d") ("e" . "f")) 
 lst1 \EV (("a" . "b") ("c" . "d") ("e" . "f")) 
 lst2 \EV (("c" . "a") ("e" . "b") ("d" . "a")) 
\endcode
\code
;; Remove all flavor names that contain "c" or "w".
 (set-difference '("strawberry" "chocolate" "banana"
                  "lemon" "pistachio" "rhubarb")
          '(#\\c #\\w)
          :test #'(lambda (s c) (find c s)))
\EV ("banana" "rhubarb" "lemon")    ;One possible ordering.
\endcode

\label Side Effects::

\funref{nset-difference} may destroy \param{list-1}.

\label Affected By:\None.

\label Exceptional Situations::

\Lazychecktypes{\param{list-1} and \param{list-2}}{\term{proper lists}}

\label See Also::

\issue{CONSTANT-MODIFICATION:DISALLOW}
{\secref\ConstantModification},
\endissue{CONSTANT-MODIFICATION:DISALLOW}
\issue{MAPPING-DESTRUCTIVE-INTERACTION:EXPLICITLY-VAGUE}
{\secref\TraversalRules}
\endissue{MAPPING-DESTRUCTIVE-INTERACTION:EXPLICITLY-VAGUE}

\label Notes::

\issue{TEST-NOT-IF-NOT:FLUSH-ALL}
The \kwd{test-not} parameter is deprecated.
\endissue{TEST-NOT-IF-NOT:FLUSH-ALL}

\endcom

%%% ========== NSET-EXCLUSIVE-OR
%%% ========== SET-EXCLUSIVE-OR
\begincom{set-exclusive-or, nset-exclusive-or}\ftype{Function}

\label Syntax::

\DefunWithValues set-exclusive-or  {list-1 list-2 {\key} key test test-not} {result-list}
\DefunWithValues nset-exclusive-or {list-1 list-2 {\key} key test test-not} {result-list}

\label Arguments and Values:: 

\param{list-1}---a \term{proper list}.

\param{list-2}---a \term{proper list}.

\param{test}---a \term{designator} for a \term{function} of two \term{arguments}
  that returns a \term{generalized boolean}.

\param{test-not}---a \term{designator} for 
  a \term{function} of two \term{arguments}
  that returns a \term{generalized boolean}.

\param{key}---a \term{designator} for a \term{function} of one argument,
  or \nil.

\param{result-list}---a \term{list}.

\label Description::
%% 15.5.0 22
\funref{set-exclusive-or} returns a \term{list} of elements that appear
in exactly one of \param{list-1} and \param{list-2}.

%% 15.5.0 25
\funref{nset-exclusive-or} 
is the \term{destructive} version of \funref{set-exclusive-or}.

%% 15.5.0 24
For all possible ordered pairs consisting of
one element from \param{list-1} and one element from \param{list-2}, the 
\kwd{test} or \kwd{test-not} function is
used to determine whether they \term{satisfy the test}.  

If \kwd{key} is supplied, it is used to
extract the part to be tested from the \param{list-1} or \param{list-2} element. 
The first argument to the \kwd{test} or \kwd{test-not} function 
is the part of an element of \param{list-1} extracted by the \kwd{key}
function (if supplied); the second argument  is the part of an 
element of \param{list-2} extracted by the \kwd{key} function (if supplied).
If \kwd{key} is not supplied or \nil, the \param{list-1} or
\param{list-2} element is used.

The result contains precisely
those elements of \param{list-1} and \param{list-2} 
that appear in no matching pair.

The result \term{list} of \funref{set-exclusive-or} 
might share storage with one of \param{list-1} or \param{list-2}.

\label Examples::

\code
 (setq lst1 (list 1 "a" "b")
       lst2 (list 1 "A" "b")) \EV (1 "A" "b")
 (set-exclusive-or lst1 lst2) \EV ("b" "A" "b" "a")
 (set-exclusive-or lst1 lst2 :test #'equal) \EV ("A" "a")
 (set-exclusive-or lst1 lst2 :test 'equalp) \EV NIL 
 (nset-exclusive-or lst1 lst2) \EV ("a" "b" "A" "b") 
 (setq lst1 (list (("a" . "b") ("c" . "d") ("e" . "f"))))
\EV (("a" . "b") ("c" . "d") ("e" . "f"))
 (setq lst2 (list (("c" . "a") ("e" . "b") ("d" . "a"))))
\EV (("c" . "a") ("e" . "b") ("d" . "a")) 
 (nset-exclusive-or lst1 lst2 :test #'string= :key #'cdr)
\EV (("c" . "d") ("e" . "f") ("c" . "a") ("d" . "a")) 
 lst1 \EV (("a" . "b") ("c" . "d") ("e" . "f"))
 lst2 \EV (("c" . "a") ("d" . "a")) 
\endcode

\label Side Effects::

\issue{REMF-DESTRUCTION-UNSPECIFIED:X3J13-MAR-89}
\funref{nset-exclusive-or} is permitted to modify any part,
\term{car} or \term{cdr}, of the \term{list structure} of \param{list-1} or \param{list-2}.
\endissue{REMF-DESTRUCTION-UNSPECIFIED:X3J13-MAR-89}

\label Affected By:\None.

\label Exceptional Situations::

\Lazychecktypes{\param{list-1} and \param{list-2}}{\term{proper lists}}

\label See Also::

\issue{CONSTANT-MODIFICATION:DISALLOW}
{\secref\ConstantModification},
\endissue{CONSTANT-MODIFICATION:DISALLOW}
\issue{MAPPING-DESTRUCTIVE-INTERACTION:EXPLICITLY-VAGUE}
{\secref\TraversalRules}
\endissue{MAPPING-DESTRUCTIVE-INTERACTION:EXPLICITLY-VAGUE}

\label Notes::

\issue{TEST-NOT-IF-NOT:FLUSH-ALL}
The \kwd{test-not} parameter is deprecated.
\endissue{TEST-NOT-IF-NOT:FLUSH-ALL}

\issue{REMF-DESTRUCTION-UNSPECIFIED:X3J13-MAR-89}
Since the \funref{nset-exclusive-or} side effect is not required,
it should not be used in for-effect-only
  positions in portable code.
 
\endissue{REMF-DESTRUCTION-UNSPECIFIED:X3J13-MAR-89}

\endcom

%%% ========== SUBSETP
\begincom{subsetp}\ftype{Function}

\label Syntax::

\DefunWithValues subsetp {list-1 list-2 {\key} key test test-not} {generalized-boolean}

\label Arguments and Values::

\param{list-1}---a \term{proper list}.

\param{list-2}---a \term{proper list}.

\param{test}---a \term{designator} for a \term{function} of two \term{arguments}
  that returns a \term{generalized boolean}.

\param{test-not}---a \term{designator} for 
  a \term{function} of two \term{arguments}
  that returns a \term{generalized boolean}.

\param{key}---a \term{designator} for a \term{function} of one argument,
  or \nil.

\param{generalized-boolean}---a \term{generalized boolean}.

\label Description::

%% 15.5.0 26
\funref{subsetp} returns \term{true} if every element of \param{list-1}
\bogusterm{matches} some element of \param{list-2},
and \term{false} otherwise.

Whether a list element is the same as another list element is
determined by the functions specified by the keyword arguments.  
The first argument to the \kwd{test} or \kwd{test-not} 
function is 
%!!! Sandra: typically?
typically
part of an element of \param{list-1} extracted by
the \kwd{key} function; the second argument is  typically part of 
an element of \param{list-2} extracted by
the \kwd{key} function.

The argument to the \kwd{key} function is an element of either
\param{list-1} or \param{list-2}; the return value is part of the element
of the supplied list element.
If \kwd{key} is not supplied or \nil, 
the \param{list-1} or \param{list-2}
element itself is supplied to the \kwd{test} or \kwd{test-not} 
function.

\label Examples::

\code
 (setq cosmos '(1 "a" (1 2))) \EV (1 "a" (1 2))
 (subsetp '(1) cosmos) \EV \term{true}
 (subsetp '((1 2)) cosmos) \EV \term{false}
 (subsetp '((1 2)) cosmos :test 'equal) \EV \term{true}
 (subsetp '(1 "A") cosmos :test #'equalp) \EV \term{true}
 (subsetp '((1) (2)) '((1) (2))) \EV \term{false}
 (subsetp '((1) (2)) '((1) (2)) :key #'car) \EV \term{true}
\endcode

\label Side Effects:\None.

\label Affected By:\None.

\label Exceptional Situations::

\Lazychecktypes{\param{list-1} and \param{list-2}}{\term{proper lists}}

\label See Also::

\issue{MAPPING-DESTRUCTIVE-INTERACTION:EXPLICITLY-VAGUE}
{\secref\TraversalRules}
\endissue{MAPPING-DESTRUCTIVE-INTERACTION:EXPLICITLY-VAGUE}

\label Notes::

\issue{TEST-NOT-IF-NOT:FLUSH-ALL}
The \kwd{test-not} parameter is deprecated.
\endissue{TEST-NOT-IF-NOT:FLUSH-ALL}

\endcom

%%% ========== NUNION
%%% ========== UNION
\begincom{union, nunion}\ftype{Function}

\label Syntax::

\DefunWithValues union  {list-1 list-2 {\key} key test test-not} {result-list}
\DefunWithValues nunion {list-1 list-2 {\key} key test test-not} {result-list}

\label Arguments and Values:: 

\param{list-1}---a \term{proper list}.

\param{list-2}---a \term{proper list}.

\param{test}---a \term{designator} for a \term{function} of two \term{arguments}
  that returns a \term{generalized boolean}.

\param{test-not}---a \term{designator} for 
  a \term{function} of two \term{arguments}
  that returns a \term{generalized boolean}.

\param{key}---a \term{designator} for a \term{function} of one argument,
  or \nil.

\param{result-list}---a \term{list}.

\label Description::

\funref{union} and \funref{nunion} return a \term{list}
that contains every element that occurs in either \param{list-1} 
or \param{list-2}.                                 

%% 15.5.0 11
For all possible ordered pairs consisting of one
element from \param{list-1} 
and one element from \param{list-2}, \kwd{test} or  \kwd{test-not} is used
to determine whether they \term{satisfy the test}. 
The first argument to the \kwd{test} or \kwd{test-not} 
function is the part of the element of \param{list-1} extracted by the
\kwd{key} function (if supplied); the second argument  
is the part of the element of \param{list-2} extracted by the
\kwd{key} function (if supplied).

The argument to the \kwd{key} function is an element of
\param{list-1} or \param{list-2}; the return value is part of the supplied
element.  
If \kwd{key} is not supplied or \nil, 
the element of \param{list-1} or \param{list-2}
itself is supplied to the \kwd{test} or \kwd{test-not} function. 

For every matching pair, 
%Removed per barmar -kmp 29-Jul-91
%at least
one of the two elements of the pair will be in the result.  Any
element from either \param{list-1} or \param{list-2}
that matches no element of the other will appear
in the result.  

%% 15.5.0 9
If there is a duplication between \param{list-1} 
and \param{list-2},
only one of the duplicate instances will be in the result.
If either \param{list-1} 
or \param{list-2} has duplicate entries within it,
the redundant entries
might or might not appear in the result.

%% 15.5.0 10
The order of elements in the result do not have to
reflect the ordering of \param{list-1} or \param{list-2} in any way.
The result \term{list} may be \funref{eq} to either
\param{list-1} or \param{list-2} if appropriate.

\label Examples::

\code
 (union '(a b c) '(f a d))
\EV (A B C F D)
\OV (B C F A D)
\OV (D F A B C)
 (union '((x 5) (y 6)) '((z 2) (x 4)) :key #'car)
\EV ((X 5) (Y 6) (Z 2))
\OV ((X 4) (Y 6) (Z 2))

 (setq lst1 (list 1 2 '(1 2) "a" "b")
       lst2 (list 2 3 '(2 3) "B" "C"))
\EV (2 3 (2 3) "B" "C")
 (nunion lst1 lst2)
\EV (1 (1 2) "a" "b" 2 3 (2 3) "B" "C") 
\OV (1 2 (1 2) "a" "b" "C" "B" (2 3) 3)
\endcode

\label Side Effects::

\issue{REMF-DESTRUCTION-UNSPECIFIED:X3J13-MAR-89}
\funref{nunion} is permitted to modify any part, \term{car} or \term{cdr}, 
of the \term{list structure} of \param{list-1} or \param{list-2}.
\endissue{REMF-DESTRUCTION-UNSPECIFIED:X3J13-MAR-89}

\label Affected By:\None.

\label Exceptional Situations::

\Lazychecktypes{\param{list-1} and \param{list-2}}{\term{proper lists}}

\label See Also::

\funref{intersection},
\issue{CONSTANT-MODIFICATION:DISALLOW}
{\secref\ConstantModification},
\endissue{CONSTANT-MODIFICATION:DISALLOW}
\issue{MAPPING-DESTRUCTIVE-INTERACTION:EXPLICITLY-VAGUE}
{\secref\TraversalRules}
\endissue{MAPPING-DESTRUCTIVE-INTERACTION:EXPLICITLY-VAGUE}

\label Notes::

\issue{TEST-NOT-IF-NOT:FLUSH-ALL}
The \kwd{test-not} parameter is deprecated.
\endissue{TEST-NOT-IF-NOT:FLUSH-ALL}

Since the \funref{nunion} side effect is not required,
it should not be used in for-effect-only positions in portable code.

\endcom

